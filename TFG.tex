\RequirePackage[l2tabu]{nag}
\documentclass[a4paper]{article}
\usepackage[T1]{fontenc}
\usepackage[utf8]{inputenc}
\usepackage{amsmath}
\usepackage{amssymb}
\usepackage{amsthm}
\usepackage{lmodern}
\usepackage{microtype}
\usepackage{tikz}
\usetikzlibrary{cd}

\theoremstyle{plain}
\newtheorem{theorem}{Theorem}[section]
\newtheorem{proposition}[theorem]{Proposition}
\newtheorem{lemma}[theorem]{Lemma}
\newtheorem{corollary}[theorem]{Corollary}
\newtheorem{example}[theorem]{Example}

\theoremstyle{definition}
\newtheorem{definition}[theorem]{Definition}

\DeclareMathOperator{\Obj}{Obj}

\newcommand{\cat}[1]{\mathcal{#1}}

\title{Monads in Haskell}
%\title{How~Haskell remains~purely~functional using~category~theory}
\author{Claudi Lleyda Moltó}
\date{}

\begin{document}
\maketitle
\tableofcontents
\clearpage
\section{Monads in Category Theory}
\subsection{Introduction to Category Theory}
\begin{definition}[Category]
\end{definition}
\begin{definition}[Functor]
\end{definition}
\begin{definition}[Natural transformation]
\end{definition}
\subsection{Monads}
\begin{definition}[Monad]
\end{definition}
\begin{example}[Pointed set monad]
\end{example}
\begin{example}[Words monad]
\end{example}

\section{The Kleisli Category in Computer Science}
\subsection{Motivation}
\begin{example}[Partiality]
\end{example}
\begin{example}[Non-determinism]
\end{example}
\begin{example}[Side-effects]
\end{example}
\begin{example}[Exceptions]
\end{example}

\subsection{Kleisli Triple}
\begin{definition}[Kleisli Triple]
    %A \emph{Kleisli Triple} over a category~\(\cat{C}\)
    %is a triple~\((T, \mu, (-)^{\ast})\),
    %consisting of
    %\begin{enumerate}
        %\item a class function
            %\(T\colon\Obj(\cat{C})\longrightarrow\Obj(\cat{C})\).
        %\item \(\forall A\in\cat{C}\)
            %a morphism~\(\eta_{A}\colon A\longrightarrow TA\)
            %in~\(\cat{C}\).
        %\item \(\forall f\colon A\longrightarrow TB\)
            %a morphism~\(f^{\ast}\colon TA\longrightarrow TB\).
    %\end{enumerate}
    %such that
    %\begin{enumerate}
        %\item \(\forall A\in\cat{C}\) we have
            %\(\eta^{\ast}_{TA} = 1_{TA}\colon TA\longrightarrow TA\).
        %\item \(\forall A,B\in\cat{C}\)
            %and \(\forall f\colon A\longrightarrow TB\)
            %the diagram
            %\[\begin{tikzcd}
                %TA \arrow[r, "f^{\ast}"] & TB \\
                %A \arrow[u, "\eta_{A}"] \arrow[ur, "f", swap] &
            %\end{tikzcd}\]
            %commutes. This is~\(f^{\ast}\circ\eta_{A} = f\).
        %\item \(g^{\ast}\circ f^{\ast} = (g^{\ast}\circ f)^{\ast}\)
    %\end{enumerate}
\end{definition}

\begin{proposition}
    There is a bijective correspondence between Kleisli triples and monads.
\end{proposition}
%\begin{proof}
    %Let~\((T, \mu, (-)^{\ast})\) be a Kleisli triple
    %and set
    %\begin{align*}
        %T\colon \cat{C} & \longrightarrow\cat{C} \\
        %C & \longmapsto TC \\
        %f\colon C\rightarrow C' & \longmapsto
        %(\eta_{C'}\circ f)^{\ast}\colon TC\longrightarrow TC'
    %\end{align*}
    %and
    %\[
        %\mu_{A} = (1_{TA})^{\ast}.
    %\]

    %Let's check that~\(T\) is a functor.
    %For every~\(f\colon C_{1}\longrightarrow C_{2}\)
    %and~\(g\colon C_{2}\longrightarrow C_{3}\)
    %\begin{align*}
        %Tg\circ Tf &= (\eta_{C_{3}}\circ g)^{\ast}\circ
                      %(\eta_{C_{2}}\circ f)^{\ast} \\
                   %&= \bigl((\eta_{C_{3}}\circ g)^{\ast}\circ
                      %(\eta_{C_{2}}\circ f)\bigr)^{\ast} \\
                   %&= \bigl((\eta_{C_{3}}\circ g)^{\ast}\circ
                      %\eta_{C_{2}}\circ f\bigr)^{\ast} \\
                   %&= (\eta_{C_{3}}\circ g\circ f)^{\ast} \\
                   %&= T(g\circ f)
    %\end{align*}
    %and for all~\(c\in\cat{C}\),
    %\[
        %T1_{C} = (\eta_{C}\circ 1_{C})^{\ast} = \eta_{C}^{\ast} = 1_{TC}.
    %\]
%\end{proof}

\section{The Kleisli Category and Monads}
\subsection{Adjuntions}
\begin{definition}[Adjuction]
\end{definition}
\begin{proposition}
    Equivalent definitions to adjunction.
\end{proposition}
\begin{proposition}
    Every adjunction induces a monad.
\end{proposition}
\begin{proposition}
    Every monad induces an adjunction.
\end{proposition}
\subsection{Free $\mathbb{T}$-algebras}
\begin{definition}[Free $\mathbb{T}$-algebras]
\end{definition}
\subsection{The Kleisli Category}
\begin{proposition}
    That statement on free $\mathbb{T}$-algebras and the Kleisli Category.
\end{proposition}

\section{Monads in Haskell}
\subsection{Haskell}
\subsection{The Monad Typeclass}
\subsection{Haskell Monads are Monads}
\subsection{Kleisli triples are Haskell Monads}

\end{document}

