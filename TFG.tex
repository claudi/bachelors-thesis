\RequirePackage[l2tabu]{nag}
\documentclass[a4paper]{article}
\usepackage[T1]{fontenc}
\usepackage[utf8]{inputenc}
\usepackage{amsmath}
\usepackage{amssymb}
\usepackage{amsthm}
\usepackage{lmodern}
\usepackage{microtype}
\usepackage{tikz}
\usetikzlibrary{cd}
\usepackage{biblatex}
\bibliography{refs.bib}
\usepackage{minted}
\usepackage[hidelinks, pdfencoding=auto]{hyperref}

\theoremstyle{plain}
\newtheorem{theorem}{Theorem}[section]
\newtheorem{proposition}[theorem]{Proposition}
\newtheorem{lemma}[theorem]{Lemma}
\newtheorem{corollary}[theorem]{Corollary}
\newtheorem{example}[theorem]{Example}

\theoremstyle{definition}
\newtheorem{definition}[theorem]{Definition}
\newtheorem{notation}[theorem]{Notation}

\DeclareMathOperator{\Obj}{Obj}
\DeclareMathOperator{\Hom}{Hom}

\newcommand{\id}{\mathrm{id}}
\newcommand{\Set}{\mathrm{Set}}

\newcommand{\cat}[1]{\mathcal{#1}}

\newcommand{\ie}{{i}.{e}., }
\newcommand{\eg}{{e}.{g}. }

\title{Monads in Haskell}
\author{Claudi Lleyda Moltó}
\date{}

\begin{document}
\maketitle
\begin{abstract}
    Upon observing that applying a certain restriction to functions in
    programming, we obtain some substantial benefits but also some important
    apparent limitations, we introduce an abstraction through category theory
    that seems to solve all of the deficits, leaving us in a ``all of the good,
    none of the bad'' situation. We also see how this abstraction can be
    successfully implemented in a real-life programming language, namely
    Haskell.
\end{abstract}
\tableofcontents
\clearpage

\section{Motivation}
\label{sec:motivation}
We begin with an observation: functions found in computer programming may not
behave like functions in mathematics.

Unfortunately, it is common to call computer functions and mathematical
functions both functions. In this report we will do so, as it is usually clear
by the context, and only specifying when there might be confusion. In these
situations, we may refer to a computer function as a~\emph{procedure}.

An example of this is the \texttt{malloc()} function found in
the~\texttt{stdlib.h} library in the~\texttt{C} programming language, which
takes a size in number of bits as input and outputs an unique pointer to an
address to a block of memory allocated to fit the requested size, which is
marked as reserved. This function

\begin{enumerate}
    \item generally outputs a different pointer on calls with the same input,
    \item modifies global state by marking allocated memory as reserved.
\end{enumerate}

These properties are incompatible with the definition of a mathematical
function. The first one means it is not well defined, and the second one means
it has side effects, which mathematical functions do not have. This motivates
the following definitions:

\begin{definition}[Pure function]
    We say that a procedure that
    \begin{enumerate}
        \item outputs the same result for the same inputs (\ie is well defined),
        \item does not modify global state.
    \end{enumerate}
    is~\emph{pure}.
    Conversely, we say a procedure that is not pure is~\emph{impure}.
\end{definition}

So, the~\texttt{malloc()} function is impure, and a function that outputs the
result of squaring an integer is pure.

In theory, a compiler could use its knowledge of certain functions being pure to
better reason about code, which can be used to obtain benefits such as

\begin{enumerate}
    \item If a pure function is called more than once with the
        same input, the result calculated on the previous run can be reused
        without having to call the function again~\cite{frostMemorization}. This
        is called~\emph{memorization}

    \item Since pure functions have no side-effects, it is usually safe to call
        them in parallel \cite{SussPureFunctionParallelization}, leading to
        great optimization.
\end{enumerate}

While pure functions present some significant benefits, they also come with some
apparent deficits. For example, it is not obvious how pure functions could react
to user input or output anything to the screen, such as debugging information or
logs. They also seem to be limited to not using global variables.

Most languages circumvent these problems by providing an interface to define
both pure and impure functions. In this way, some of the benefits can be
applied, and the deficits can be handled by impure functions.

As of this point, the problems we mentioned are only apparent, and while it is
not at all obvious how they could be avoided, there is also no definitive proof
they cannot. We promise to provide an abstraction that solves these limitations,
and then we will see how it can be successfully implemented into a useful
programming language.

\section{Monads in Category Theory}
\subsection{Introduction to Category Theory}

\begin{definition}[Category]
    \label{def:category}
    A \emph{category}~\(\cat{C}\) is composed of a collection~\(\Obj(\cat{C})\)
    of \emph{objects}, for every two objects~\(A,B\in\Obj(\cat{C})\) a
    set~\(\Hom_{\cat{C}}(A,B)\) of \emph{morphisms} and a \emph{composition
    function}
    \[
        \circ:\Hom_{\cat{C}}(A,B)\times\Hom_{\cat{C}}(B,C) \longrightarrow
        \Hom_{\cat{C}}(A,C)
    \]
    that satisfies
    \begin{enumerate}
        \item \emph{Associativity}: for every three
            morphisms~\(f\in\Hom_{\cat{C}}(A,B)\),~\(g\in\Hom_{\cat{C}}(B,C)\)
            and~\(h\in\Hom_{\cat{C}}(C,D)\)
            we have
            \[
                h \circ (g \circ f) = (h \circ g) \circ f.
            \]
        \item \emph{Left and right units}: for every
            object~\(A\in\Obj(\cat{C})\) there exists a unique
            element~\(\id_{A}\in\Hom_{\cat{C}}(A,A)\)
            such that for every morphism~\(f\in\Hom_{\cat{C}}(A,B)\) we have
            \[
                f \circ \id_{B} = f,
            \]
            and for every morphism~\(g\in\Hom_{\cat{C}}(B,A)\) we have
            \[
                \id_{A} \circ g = g.
            \]
    \end{enumerate}
\end{definition}

\begin{notation}
    We write~\(f:A\longrightarrow B\) or~\(A\overset{f}{\longrightarrow}B\) to
    mean~\(f\in\Hom_{\cat{C}}(A,B)\) and~\(gf\) to mean~\(g\circ f\).
\end{notation}

\begin{example}[Category of sets]
    \label{cat:set}
    There is a category~\(\Set\), where the objects are sets and morphisms are
    maps between sets.
\end{example}

\begin{definition}[Functor]
    \label{def:functor}
    A~\emph{functor}~\(T\) from a category~\(\cat{C}\) to a category~\(\cat{D}\)
    is a map~\(T:\Obj(\cat{C})\longrightarrow\Obj(\cat{D})\), and for every
    pair of objects~\(A,B\in\Obj{C}\) a
    map~\(T:\Hom_{\cat{C}}(A,B)\longrightarrow\Hom_{\cat{D}}(TA,TB)\),
    such that~\(T\) preserves
    \begin{enumerate}
        \item \emph{Composition}: for every two
            morphisms~\(f:A\longrightarrow B\),~\(g:B\longrightarrow C\)
            in~\(\cat{C}\) we have
            \[
                T(g \circ f) = T(g) \circ T(f).
            \]
        \item \emph{Units}: for each object~\(A\in\Obj{C}\) we have
            \[
                T(\id_{A}) = \id_{T(A)}.
            \]
    \end{enumerate}
\end{definition}

\begin{notation}
    We write~\(T:\cat{C}\longrightarrow\cat{D}\) to mean~\(T\) is a functor from
    a category~\(\cat{C}\) to a category~\(\cat{D}\).
    We may also shorten~\(T(A)\) to~\(TA\) or~\(T(f)\) to~\(Tf\), given~\(A\) an
    object in~\(\cat{C}\) and~\(f\) a morphism in~\(\cat{C}\).
\end{notation}

\begin{definition}[Natural transformation]
    \label{def:natural-transformation}
    Let~\(S,T:\cat{C}\longrightarrow\cat{D}\) be functors. A \emph{natural
    transformation}~\(\tau:S\longrightarrow T\) is a
    family~\(\tau=\{\tau_{A}:SA\longrightarrow TA\}_{A\in\Obj(\cat{C})}\) of
    morphisms in~\(\cat{D}\) such that for every
    morphism~\(f:A\longrightarrow B\) in~\(\cat{C}\)
    \[
        Tf \circ \tau_{A} = \tau_{B} \circ Sf,
    \]
    that is, making the following diagram commutative
    \[
        \begin{tikzcd}
            SA \arrow[r, "\tau_{A}"] \arrow[d, "Sf"] & TA \arrow[d, "Tf"] \\
            SB \arrow[r, "\tau_{B}"] & TB
        \end{tikzcd}
    \]
\end{definition}

\subsection{Monads}
\begin{definition}[Monad]
    \label{def:monad}
    A~\emph{monad} over a category~\(\cat{C}\) is a triple~\((T,\eta,\mu)\)
    where~\(T:\cat{C}\longrightarrow\cat{C}\) is a functor
    and~\(\eta:\id\longrightarrow T\) and~\(\mu:TT\longrightarrow T\) are two
    natural transformations such that for every object~\(A\in\Obj(\cat{C})\) the
    diagrams
    \[
        \begin{tikzcd}
            TA \arrow[r, "T\eta_{A}"] & TTA \arrow[d, "\mu_{A}"] & TA \arrow[l,
            swap, "\eta_{TA}"] \\
                                      & TA \arrow[ul, equal] \arrow[ur, equal] &
        \end{tikzcd}
        \qquad\text{and}\qquad
        \begin{tikzcd}
            TTTA \arrow[r, "\mu_{TA}"] \arrow[d, "T\mu_{A}"] & TTA \arrow[d,
            "\mu_{A}"] \\
            TTA \arrow[r, "\mu_{A}"] & TA
        \end{tikzcd}
    \]
    are commutative. This is
    \[
        \mu_{A}\circ T\eta_{A} = \id_{TA} = \mu_{A}\circ\eta_{TA}
        \qquad\text{and}\qquad
        \mu_{A}\circ \mu_{TA}
        = \mu_{A} \circ T\mu_{A}.
    \]
    We call~\(\eta\) and~\(\mu\) the \emph{unit} and the \emph{multiplication}
    of the monad, respectively.
\end{definition}

\begin{example}[Pointed set monad]
    \label{monad:maybe}
    We define the pointed set monad~\((M,\eta,\mu)\) on the category of
    sets as follows.
    \begin{enumerate}
        \item The functor~\(M\) for any object~\(A\) or
            morphism~\(f:A\longrightarrow B\) is defined as
            \begin{gather*}
                MA = A\sqcup\{\bot\}
                \qquad\text{and}\qquad
                \begin{split}
                    Mf:MA&\longrightarrow MB \\
                    \bot&\longmapsto\bot \\
                    A\ni x&\longmapsto f(x)
                \end{split}
            \end{gather*}
        \item The unit~\(\eta\) for any object~\(A\) is defined as the inclusion
            of~\(A\) into~\(MA\).
        \item The multiplication~\(\mu\) for any object~\(A\) is defined as
            \begin{align*}
                \mu_{A}:MMA&\longrightarrow MA \\
                \bot_{1}&\longmapsto \bot \\
                \bot_{2}&\longmapsto \bot \\
                A\ni x&\longmapsto x
            \end{align*}
            where we have denoted~\(MMA=A\sqcup\{\bot_{1}\}\sqcup\{\bot_{2}\}\)
            and~\(MA=A\sqcup\{\bot\}\) for convenience.
    \end{enumerate}
\end{example}

\begin{example}[Power set monad]
    \label{monad:power-set}
    We define the power set monad~\((P,\eta,\mu)\) on the category of sets as
    follows.
    \begin{enumerate}
        \item The functor~\(P\) for any object~\(A\) or
            morphism~\(f:A\longrightarrow B\) is defined as
            \begin{gather*}
                PA = \mathcal{P}(A)
                \qquad\text{and}\qquad
                \begin{split}
                    Pf:PA&\longrightarrow PB \\
                    S&\longmapsto\{f(x)\in B \mid x\in S\}
                \end{split}
            \end{gather*}
        \item The unit~\(\eta\) for any object~\(A\) is defined
            as~\(\eta_{A}(x)=\{x\}\).
        \item The multiplication~\(\mu\) for any object~\(A\) is defined as
            \begin{align*}
                \mu_{A}:PPA&\longrightarrow PA \\
                S&\longmapsto \bigcup_{X\in S}X
            \end{align*}
    \end{enumerate}
\end{example}

\section{The Kleisli Triples in Computer Science}
In section~\ref{sec:motivation} we established that pure functions present some
deficits. Let's put category theory aside and see how we could work around some
of these, which we will then generalize through categorical notions.
\begin{example}[Non-determinism]
\end{example}
\begin{example}[Side-effects]
\end{example}

While not being problems inherent to pure functions, there are other situations
commonly found in computer programming that are also foreign to mathematical
functions, which pure functions behave like.

For example, it is difficult to specify the domain and codomain of a procedure,
which often leads to functions that are not defined on all of their input. An
instance of this is the factorial function, which may be written as
\begin{minted}{python}
def factorial(n: int):
    if n == 0:
        return 1
    elif n >= 1:
        return n * factorial(n - 1)
\end{minted}
This implementation of the factorial function accepts an integer as its input,
but it is not defined for negative values. We call this~\emph{partiality}.

\begin{example}[Partiality]
\end{example}

Another example are~\emph{exceptions}. Procedures often encounter errors during
their execution, which mathematical functions do not, and must signal them to
other functions that have to deal with them, either by passing the exception up
the call chain (eventually aborting the program) or by directly dealing with it
during their execution.

\begin{example}[Exceptions]
\end{example}

\subsection{Kleisli Triple}
\begin{definition}[Kleisli Triple]
    We define a \emph{Kleisli Triple} over a category~\(\cat{C}\)
    as a triple~\((T, \eta, (-)^{\ast})\) consisting of
    \begin{enumerate}
        \item a class function
            \(T:\Obj(\cat{C})\longrightarrow\Obj(\cat{C})\).
        \item \(\forall A\in\cat{C}\)
            a morphism~\(\eta_{A}:A\longrightarrow TA\)
            in~\(\cat{C}\).
        \item \(\forall f:A\longrightarrow TB\)
            a morphism~\(f^{\ast}:TA\longrightarrow TB\).
    \end{enumerate}
    such that
    \begin{enumerate}
        \item For all~\(A\in\cat{C}\) we have
            \(\eta^{\ast}_{TA} = \id_{TA}:TA\longrightarrow TA\).
        \item For all~\(A,B\in\cat{C}\)
            and \(f:A\longrightarrow TB\)
            the diagram
            \[\begin{tikzcd}
                TA \arrow[r, "f^{\ast}"] & TB \\
                A \arrow[u, "\eta_{A}"] \arrow[ur, "f", swap] &
            \end{tikzcd}\]
            commutes. That is~\(f^{\ast}\circ\eta_{A} = f\).
        \item For all~\(f:A\to TB,g:B\to TC\) we
            have~\(g^{\ast}\circ f^{\ast} = (g^{\ast}\circ f)^{\ast}\).
    \end{enumerate}
\end{definition}

\begin{proposition}
    Let~\(\cat{C}\) be a category.
    There is a bijective correspondence between Kleisli Triples over~\(\cat{C}\)
    and monads over~\(\cat{C}\).
\end{proposition}
\begin{proof}
    Let~\((T, \eta, (-)^{\ast})\) be a Kleisli triple over~\(\cat{C}\) and set
    \begin{equation}
        \label{eq:monad-unit-in-kleisli-trip}
        \begin{split}
            T:\cat{C} & \longrightarrow\cat{C} \\
            A & \longmapsto TA \\
            f:A\rightarrow B & \longmapsto
            (\eta_{B}\circ f)^{\ast}:TA\longrightarrow TB
        \end{split}
    \end{equation}
    and
    \begin{equation}
        \label{eq:monad-prod-in-kleisli-trip}
        \mu_{A} = (\id_{TA})^{\ast}.
    \end{equation}
    We will see that~\((T,\eta,\mu)\) is a monad.

    Let's check that~\(T\) is a functor.
    For every two~\(f:A\longrightarrow B\)
    and~\(g:B\longrightarrow C\)
    morphisms in~\(\cat{C}\)
    we have
    \begin{align*}
        Tg\circ Tf &= (\eta_{C}\circ g)^{\ast}\circ
                      (\eta_{B}\circ f)^{\ast} \\
                   &= \bigl((\eta_{C}\circ g)^{\ast}\circ
                      (\eta_{B}\circ f)\bigr)^{\ast} \\
                   &= \bigl((\eta_{C}\circ g)^{\ast}\circ
                      \eta_{B}\circ f\bigr)^{\ast} \\
                   &= (\eta_{C}\circ g\circ f)^{\ast} \\
                   &= T(g\circ f)
    \end{align*}
    and for all~\(A\in\cat{C}\),
    \[
        T\id_{A} = (\eta_{A}\circ \id_{A})^{\ast} = \eta_{A}^{\ast} = \id_{TA},
    \]
    which proves that~\(T\) is a functor.

    Next we want to see that~\(\mu=\{\mu_{A}\}_{A\in\Obj(\cat{C})}\) is a
    natural transformation.
    Given morphism~\(f:A\longrightarrow B\) in~\(\cat{C}\) we want to prove that
    \[
        Tf \circ \mu_{A} = \mu_{B} \circ TTf.
    \]
    Using~\eqref{eq:monad-prod-in-kleisli-trip} we get
    \begin{gather*}
        \begin{split}
            Tf \circ \mu_{A} &= Tf \circ (\id_{TA})^{\ast} \\
                &= (\eta_{B} \circ f)^{\ast} \circ (\id_{TA})^{\ast} \\
                &= \bigl((\eta_{B} \circ f)^{\ast} \circ \id_{TA}\bigr)^{\ast} \\
                &= \bigl((\eta_{B} \circ f)^{\ast}\bigr)^{\ast} \\
                &= (Tf)^{\ast}
        \end{split}
        \qquad\text{and}\qquad
        \begin{split}
            \mu_{B} \circ TTf &= (\id_{TB})^{\ast} \circ TTf \\
                &= (\id_{TB})^{\ast} \circ (\eta_{TB} \circ Tf)^{\ast} \\
                &= \bigl((\id_{TB})^{\ast} \circ \eta_{TB} \circ Tf\bigr)^{\ast} \\
                &= \bigl(\id_{TB} \circ Tf\bigr)^{\ast} \\
                %&= \bigl(T\id_{B} \circ Tf\bigr)^{\ast} \\
                &= (Tf)^{\ast}
    \end{split}
    \end{gather*}
    and we have proved that~\(\mu\) is a natural transformation.

    Finally, we can check that~\(\eta=\{\eta_{A}\}_{A\in\Obj(\cat{C})}\) is also
    a natural transformation.
    By using~\eqref{eq:monad-unit-in-kleisli-trip} we can see that, for any
    morphism~\(f:A\longrightarrow B\) in~\(\cat{C}\), we have
    \[
        Tf \circ \eta_{A}
        = (\eta_{B}\circ f)^{\ast} \circ \eta_{A}
        = \eta_{B} \circ f,
    \]
    and this proves that a Kleisli Triple induces a monad.

    Let~\((T,\eta,\mu)\) be a monad over~\(\cat{C}\) and for any
    morphism~\(f:A\longrightarrow TB\) in~\(\cat{C}\) set
    \begin{equation}
        \label{eq:kleisli-ast-in-monad}
        f^{\ast} = \mu_{B} \circ Tf.
    \end{equation}
    We want to check that~\((T,\eta,(-)^{\ast})\) is a Kleisli triple.

    By the monad unit law we have
    \[
        \eta_{TA}^{\ast} = \mu_{TA} \circ T\eta_{TA} = \id_{TA},
    \]
    and for any two morphisms~\(f:A\longrightarrow TB\)
    and~\(g:B\longrightarrow TC\) in~\(\cat{C}\) we have
    \begin{align*}
        g^{\ast}\circ f^{\ast} &= \mu_{C} \circ Tg \circ \mu_{B} \circ Tf \\
                               &= \mu_{C} \circ \mu_{TC} \circ TTg \circ Tf \\
                               &= \mu_{C} \circ T\mu_{C} \circ TTg \circ Tf \\
                               &= \mu_{C} \circ T(\mu_{C} \circ Tg \circ f) \\
                               &= \mu_{C} \circ T(g^{\ast} \circ f) \\
                               &= (g^{\ast} \circ f)^{\ast}
    \end{align*}
    and this proves that every monad induces a Kleisli Triple.

    It is clear that these two constructions are inverse to each other.
\end{proof}

\section{Monads in Haskell}
\subsection{Haskell}
\subsection{The monad typeclass}
The Haskell language allows users to define~\emph{typeclasses}, which area set
of constraints we give to a certain set of types.
They are similar to interfaces, which appear in other languages.

For our purposes, we can picture them as a definition. A typeclass describes a
behavior we expect.

Let's see an example. The Functor typeclass is
\begin{minted}{Haskell}
class Functor f where
    fmap :: (a -> b) -> f a -> f b
\end{minted}
From the usage of~\texttt{f a} and~\texttt{f b}, The Haskell compiler can infer
that~\texttt{f} must be a parametric type with one parameter, and with this
information we can read the previous statement as
\begin{quote}
    For a parametric datatype~\texttt{f} with one parameter to be called a
    Functor, it must be equipped with a structure

    \texttt{fmap :: (a -> b) -> f a -> f b}
\end{quote}

Let's see an example of an implementation. If we define the parametric data
type~\texttt{Maybe} as follows
\begin{minted}{Haskell}
data Maybe a = Just a | None
\end{minted}
we can make it into a Functor by writing
\begin{minted}{Haskell}
instance Functor Maybe m where
    fmap :: (a -> b) -> m a -> m b
    fmap _ Nothing  = Nothing
    fmap f (Just x) = Just (f x)
\end{minted}

\subsection{Haskell monads are monads}
\subsection{Kleisli triples are Haskell monads}

\printbibliography

\end{document}

