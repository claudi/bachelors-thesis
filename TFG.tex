\RequirePackage[l2tabu]{nag}
\documentclass[a4paper]{article}
\usepackage[T1]{fontenc}
\usepackage[utf8]{inputenc}
\usepackage{amsmath}
\usepackage{amssymb}
\usepackage{amsthm}
\usepackage{lmodern}
\usepackage{microtype}
\usepackage{tikz}
\usetikzlibrary{cd}
\usepackage{minted}
\usepackage[hidelinks, pdfencoding=auto]{hyperref}

\theoremstyle{plain}
\newtheorem{theorem}{Theorem}[section]
\newtheorem{proposition}[theorem]{Proposition}
\newtheorem{lemma}[theorem]{Lemma}
\newtheorem{corollary}[theorem]{Corollary}
\newtheorem{example}[theorem]{Example}

\theoremstyle{definition}
\newtheorem{definition}[theorem]{Definition}
\newtheorem{notation}[theorem]{Notation}

\DeclareMathOperator{\Obj}{Obj}
\DeclareMathOperator{\Hom}{Hom}

\newcommand{\id}{\mathrm{id}}
\newcommand{\Set}{\mathrm{Set}}

\newcommand{\adjoint}{\mathrel{\vdash}}

\newcommand{\cat}[1]{\mathcal{#1}}

\title{Monads in Haskell}
\author{Claudi Lleyda Moltó}
\date{}

\begin{document}
\maketitle
\tableofcontents
\clearpage
\section{Monads in Category Theory}
\subsection{Introduction to Category Theory}

\begin{definition}[Category]
    \label{def:category}
    A \emph{category}~\(\cat{C}\) is composed of a collection~\(\Obj(\cat{C})\)
    of \emph{objects}, for every two objects~\(A,B\in\Obj(\cat{C})\) a
    set~\(\Hom_{\cat{C}}(A,B)\) of \emph{morphisms} and a \emph{composition
    function}
    \[
        \circ:\Hom_{\cat{C}}(A,B)\times\Hom_{\cat{C}}(B,C) \longrightarrow
        \Hom_{\cat{C}}(A,C)
    \]
    that satisfies
    \begin{enumerate}
        \item \emph{Associativity}: for every three
            morphisms~\(f\in\Hom_{\cat{C}}(A,B)\),~\(g\in\Hom_{\cat{C}}(B,C)\)
            and~\(h\in\Hom_{\cat{C}}(C,D)\)
            we have
            \[
                f \circ (g \circ h) = (f \circ g) \circ h.
            \]
        \item \emph{Left and right units}: for every two
            objects~\(A,B\in\Obj(\cat{C})\) there exist unique
            elements~\(\id_{A}\in\Hom_{\cat{C}}(A,A)\),~\(\id_{B}\in\Hom_{\cat{C}}(B,B)\)
            such that for every morphism~\(f\in\Hom_{\cat{C}}(A,B)\) we have
            \[
                \id_{A} \circ f = f = f \circ \id_{B}.
            \]
    \end{enumerate}
\end{definition}

\begin{notation}
    We write~\(f:A\longrightarrow B\) or~\(A\overset{f}{\longrightarrow}B\) to
    mean~\(f\in\Hom_{\cat{C}}(A,B)\) and~\(fg\) to mean~\(f\circ g\).
\end{notation}

\begin{example}[Category of sets]
    \label{cat:set}
    There is a category~\(\Set\), where the objects are sets and morphisms are
    maps between sets.
\end{example}

\begin{definition}[Functor]
    \label{def:functor}
    A~\emph{functor}~\(T\) from a category~\(\cat{C}\) to a category~\(\cat{D}\)
    is a map~\(T:\Obj(\cat{C})\longrightarrow\Obj(\cat{D})\), and for every
    morphism~\(f:A\longrightarrow B\) in~\(\cat{C}\) a
    morphism~\(T(f):T(A)\longrightarrow T(B)\) in~\(\cat{D}\), such that~\(T\)
    preserves
    \begin{enumerate}
        \item \emph{Composition}: for every two
            morphisms~\(f:A\longrightarrow B\),~\(g:B\longrightarrow C\)
            in~\(\cat{C}\) we have
            \[
                T(f \circ g) = T(f) \circ T(g).
            \]
        \item \emph{Units}: for each object~\(C\in\Obj{C}\) we have
            \[
                T(\id_{C}) = \id_{T(C)}.
            \]
    \end{enumerate}
\end{definition}

\begin{notation}
    We write~\(T:\cat{C}\longrightarrow\cat{D}\) to mean~\(T\) is a functor from
    a category~\(\cat{C}\) to a category~\(\cat{D}\).
    We may also shorten~\(T(A)\) to~\(TA\) or~\(T(f)\) to~\(Tf\), given~\(A\) an
    object in~\(\cat{C}\) and~\(f\) a morphism in~\(\cat{C}\).
\end{notation}

\begin{definition}[Natural transformation]
    \label{def:natural-transformation}
    Let~\(S,T:\cat{C}\longrightarrow\cat{D}\) be functors. A \emph{natural
    transformation}~\(\tau:S\longrightarrow T\) is a
    family~\(\tau=\{\tau_{A}:SA\longrightarrow TA\}_{A\in\Obj(A)}\) of morphisms
    in~\(\cat{B}\) such that for every morphism~\(f:A\longrightarrow B\)
    in~\(\cat{A}\)
    \[
        Tf \circ \tau_{A} = \tau_{B} \circ Sf,
    \]
    that is, making the following diagram commutative
    \[
        \begin{tikzcd}
            SA \arrow[r, "\tau_{A}"] \arrow[d, "Sf"] & TA \arrow[d, "Tf"] \\
            SB \arrow[r, "\tau_{B}"] & TB
        \end{tikzcd}
    \]
\end{definition}

\subsection{Monads}
\begin{definition}[Monad]
    \label{def:monad}
    A~\emph{monad} over a category~\(\cat{C}\) is a triple~\((T,\eta,\mu)\)
    where~\(T:\cat{C}\longrightarrow\cat{C}\) is a functor
    and~\(\eta:\id\longrightarrow T\) and~\(\mu:TT\longrightarrow T\) are two
    natural transformations such that for every object~\(A\in\Obj(\cat{C})\) the
    diagrams
    \[
        \begin{tikzcd}
            TA \arrow[r, "T\eta_{A}"] & TTA \arrow[d, "\mu_{A}"] & TA \arrow[l,
            swap, "\eta_{TA}"] \\
                                      & TA \arrow[ul, equal] \arrow[ur, equal] &
        \end{tikzcd}
        \qquad\text{and}\qquad
        \begin{tikzcd}
            TTTA \arrow[r, "\mu_{TA}"] \arrow[d, "T\mu_{A}"] & TTA \arrow[d,
            "\mu_{A}"] \\
            TTA \arrow[r, "\mu_{A}"] & TA
        \end{tikzcd}
    \]
    are commutative. This is
    \[
        \mu_{A}\circ T\eta_{A} = \id_{TA} = \mu_{A}\circ\eta_{TA}
        \qquad\text{and}\qquad
        \mu_{A}\circ \mu_{TA}
        = \mu_{A} \circ T\mu_{A}.
    \]
    We call~\(\eta\) and~\(\mu\) the \emph{unit} and the \emph{product} of the
    monad, respectively.
\end{definition}

\begin{example}[Pointed set monad]
    \label{monad:maybe}
    We define the pointed set monad~\((M,\eta,\mu)\) on the~\(\Set\) category as
    follows.
    \begin{enumerate}
        \item The functor~\(M\) for any object~\(A\) or
            morphism~\(f:A\longrightarrow B\) is defined as
            \begin{gather*}
                MA = A\sqcup\{\bot\}
                \qquad\text{and}\qquad
                \begin{split}
                    Mf:MA&\longrightarrow MB \\
                    \bot&\longmapsto\bot \\
                    A\ni x&\longmapsto f(x)
                \end{split}
            \end{gather*}
        \item The unit~\(\eta\) for any object~\(A\) is defined as the inclusion
            of~\(A\) into~\(MA\).
        \item The product~\(\mu\) for any object~\(A\) is defined as
            \begin{align*}
                \mu_{A}:MMA&\longrightarrow MA \\
                \bot_{1}&\longmapsto \bot \\
                \bot_{2}&\longmapsto \bot \\
                A\ni x&\longmapsto x
            \end{align*}
            where we've denoted~\(MMA=A\sqcup\{\bot_{1}\}\sqcup\{\bot_{2}\}\)
            and~\(MA=A\sqcup\{\bot\}\) for convenience.
    \end{enumerate}
\end{example}

\begin{example}[Power set monad]
    \label{monad:power-set}
    We define the power set monad~\((P,\eta,\mu)\) on the~\(\Set\) category as
    follows.
    \begin{enumerate}
        \item The functor~\(P\) for any object~\(A\) or
            morphism~\(f:A\longrightarrow B\) is defined as
            \begin{gather*}
                PA = \mathcal{P}(A)
                \qquad\text{and}\qquad
                \begin{split}
                    Pf:PA&\longrightarrow PB \\
                    S&\longmapsto\mathcal{P}\bigl(\{f(x)\in B \mid x\in S\}\bigr)
                \end{split}
            \end{gather*}
        \item The unit~\(\eta\) for any object~\(A\) is defined
            as~\(\eta_{A}(x)=\{x\}\).
        \item The product~\(\mu\) for any object~\(A\) is defined as
            \begin{align*}
                \mu_{A}:PPA&\longrightarrow PA \\
                S&\longmapsto \bigcup_{X\in S}X
            \end{align*}
    \end{enumerate}
\end{example}

\section{The Kleisli Triples in Computer Science}
\subsection{Motivation}
When trying to model computer functions with mathematical functions we face some
adversity, which may lead us to think that they are inherently different.

Mathematical functions must be defined on their domain, while computer functions
may not be, as the program that encapsulates them may abort during the
function's execution.
For example, the factorial mathematical function is not defined on negative
integers, but the computer function
\begin{minted}{python}
def factorial(n):
    if n == 0:
        return 1
    elif n >= 1:
        return n * factorial(n - 1)
\end{minted}
is perfectly valid, but will crash a program that tries to evaluate it on a
negative integer.
We call this a~\emph{partial} function.

Let's forget category theory and see how we could work around some of these:
\begin{example}[Partiality]
\end{example}
\begin{example}[Non-determinism]
\end{example}
\begin{example}[Side-effects]
\end{example}
\begin{example}[Exceptions]
\end{example}

\subsection{Kleisli Triple}
\begin{definition}[Kleisli Triple]
    We define a \emph{Kleisli Triple} over a category~\(\cat{C}\)
    as a triple~\((T, \eta, (-)^{\ast})\) consisting of
    \begin{enumerate}
        \item a class function
            \(T:\Obj(\cat{C})\longrightarrow\Obj(\cat{C})\).
        \item \(\forall A\in\cat{C}\)
            a morphism~\(\eta_{A}:A\longrightarrow TA\)
            in~\(\cat{C}\).
        \item \(\forall f:A\longrightarrow TB\)
            a morphism~\(f^{\ast}:TA\longrightarrow TB\).
    \end{enumerate}
    such that
    \begin{enumerate}
        \item \(\forall A\in\cat{C}\) we have
            \(\eta^{\ast}_{TA} = \id_{TA}:TA\longrightarrow TA\).
        \item \(\forall A,B\in\cat{C}\)
            and \(\forall f:A\longrightarrow TB\)
            the diagram
            \[\begin{tikzcd}
                TA \arrow[r, "f^{\ast}"] & TB \\
                A \arrow[u, "\eta_{A}"] \arrow[ur, "f", swap] &
            \end{tikzcd}\]
            commutes. This is~\(f^{\ast}\circ\eta_{A} = f\).
        \item \(g^{\ast}\circ f^{\ast} = (g^{\ast}\circ f)^{\ast}\)
    \end{enumerate}
\end{definition}

\begin{proposition}
    Let~\(\cat{C}\) be a category.
    There is a bijective correspondence between Kleisli Triples over~\(\cat{C}\)
    and monads over~\(\cat{C}\).
\end{proposition}
\begin{proof}
    Let~\((T, \eta, (-)^{\ast})\) be a Kleisli triple over~\(\cat{C}\) and set
    \begin{equation}
        \label{eq:monad-unit-in-kleisli-trip}
        \begin{split}
            T:\cat{C} & \longrightarrow\cat{C} \\
            A & \longmapsto TA \\
            f:A\rightarrow B & \longmapsto
            (\eta_{B}\circ f)^{\ast}:TA\longrightarrow TB
        \end{split}
    \end{equation}
    and
    \begin{equation}
        \label{eq:monad-prod-in-kleisli-trip}
        \mu_{A} = (\id_{TA})^{\ast}.
    \end{equation}

    Let's check that~\(T\) is a functor.
    For every two~\(f:A\longrightarrow B\)
    and~\(g:B\longrightarrow C\)
    morphisms in~\(\cat{C}\)
    we have
    \begin{align*}
        Tg\circ Tf &= (\eta_{C}\circ g)^{\ast}\circ
                      (\eta_{B}\circ f)^{\ast} \\
                   &= \bigl((\eta_{C}\circ g)^{\ast}\circ
                      (\eta_{B}\circ f)\bigr)^{\ast} \\
                   &= \bigl((\eta_{C}\circ g)^{\ast}\circ
                      \eta_{B}\circ f\bigr)^{\ast} \\
                   &= (\eta_{C}\circ g\circ f)^{\ast} \\
                   &= T(g\circ f)
    \end{align*}
    and for all~\(A\in\cat{C}\),
    \[
        T\id_{A} = (\eta_{A}\circ \id_{A})^{\ast} = \eta_{A}^{\ast} = \id_{TA},
    \]
    which proves that~\(T\) is a functor.

    Next we want to see that~\(\mu=\{\mu_{A}\}_{A\in\Obj(\cat{C})}\) is a
    natural transformation.
    Given morphism~\(f:A\longrightarrow B\) in~\(\cat{C}\) we want to prove that
    \[
        Tf \circ \mu_{A} = \mu_{B} \circ TTf.
    \]
    Expanding~\eqref{eq:monad-prod-in-kleisli-trip} we get
    \begin{gather*}
        \begin{split}
            Tf \circ \mu_{A} &= Tf \circ (\id_{TA})^{\ast} \\
                &= Tf \circ (T\id_{A})^{\ast} \\
                &= (\eta_{B} \circ f)^{\ast} \circ (T\id_{A})^{\ast} \\
                &= \bigl((\eta_{B} \circ f)^{\ast} \circ T\id_{A}\bigr)^{\ast} \\
                &= \bigl((\eta_{B} \circ f)^{\ast}\bigr)^{\ast} \\
                &= \bigl(Tf\bigr)^{\ast}
        \end{split}
        \qquad\text{and}\qquad
        \begin{split}
            \mu_{B} \circ TTf &= (\id_{TB})^{\ast} \circ TTf \\
                &= (\id_{TB})^{\ast} \circ TTf \\
                &= (\id_{TB})^{\ast} \circ (\eta_{TB} \circ Tf)^{\ast} \\
                &= \bigl((\id_{TB})^{\ast} \circ \eta_{TB} \circ Tf\bigr)^{\ast} \\
                &= \bigl(\id_{TB} \circ Tf\bigr)^{\ast} \\
                %&= \bigl(T\id_{B} \circ Tf\bigr)^{\ast} \\
                &= \bigl(Tf\bigr)^{\ast}
    \end{split}
    \end{gather*}
    and we have proved that~\(\mu\) is a natural transformation.

    Finally, we can check that~\(\eta=\{\eta_{A}\}_{A\in\Obj(\cat{C})}\) is also
    a natural transformation.
    By expanding~\eqref{eq:monad-unit-in-kleisli-trip} we can see that, for any
    morphism~\(f:A\longrightarrow B\) in~\(\cat{C}\), we have
    \[
        Tf \circ \eta_{A}
        = (\eta_{B}\circ f)^{\ast} \circ \eta_{A}
        = \eta_{B} \circ f,
    \]
    and this proves that a Kleisli Triple induces a monad.

    Let~\((T,\eta,\mu)\) be a monad over~\(\cat{C}\) and for any
    morphism~\(f:A\longrightarrow TB\) in~\(\cat{C}\) set
    \begin{equation}
        \label{eq:kleisli-ast-in-monad}
        f^{\ast} = \mu_{B} \circ Tf.
    \end{equation}
    By the monad unit law we have
    \[
        \eta_{TA}^{\ast} = \mu_{TA} \circ T\eta_{TA} = \id_{TA},
    \]
    and for any two morphisms~\(f:A\longrightarrow TB\)
    and~\(g:B\longrightarrow TC\) in~\(\cat{C}\) we have
    \begin{align*}
        g^{\ast}\circ f^{\ast} &= \mu_{C} \circ Tg \circ \mu_{B} \circ Tf \\
                               &= \mu_{C} \circ \mu_{TC} \circ TTg \circ Tf \\
                               &= \mu_{C} \circ T\mu_{C} \circ TTg \circ Tf \\
                               &= \mu_{C} \circ T(\mu_{C} \circ Tg \circ f) \\
                               &= \mu_{C} \circ T(g^{\ast} \circ f) \\
                               &= (g^{\ast} \circ f)^{\ast}
    \end{align*}
    and this proves that every monad induces a Kleisli Triple.
\end{proof}

\section{Monads in Haskell}
We begin with an observation: functions found in computer programming may not
behave like functions in mathematics; they are not mathematical functions.

An example is the \texttt{malloc()} function found in the~\texttt{stdlib}
library in the~\texttt{C} programming language, which takes a size in number of
bits and outputs an unique pointer to an address to memory allocated to fit the
requested size. This function

\begin{enumerate}
    \item generally outputs a different pointer on calls with the same input,
    \item modifies global state by marking allocated memory as being used.
\end{enumerate}

These properties are incompatible with the definition of a mathematical
function. The first one means it is not well defined, and the second one means
it has side effects, which mathematical functions do not have.

\subsection{Haskell}
\subsection{The monad typeclass}
\subsection{Haskell monads are monads}
\subsection{Kleisli triples are Haskell monads}

\end{document}

