\RequirePackage[l2tabu]{nag}
\documentclass[a4paper]{article}
\usepackage[T1]{fontenc}
\usepackage[utf8]{inputenc}
\usepackage{amsmath}
\usepackage{amssymb}
\usepackage{amsthm}
\usepackage{lmodern}
\usepackage{microtype}

\theoremstyle{plain}
\newtheorem{theorem}{Theorem}[section]
\newtheorem{proposition}[theorem]{Proposition}
\newtheorem{lemma}[theorem]{Lemma}
\newtheorem{corollary}[theorem]{Corollary}

\theoremstyle{definition}
\newtheorem{definition}[theorem]{Definition}

\begin{document}
\tableofcontents
\clearpage
\section{Monads in Category Theory}
\subsection{Introduction to Category Theory}
\begin{definition}[Category]
\end{definition}
\begin{definition}[Functor]
\end{definition}
\begin{definition}[Natural transformation]
\end{definition}
\subsection{Monads}

\section{The Kleisli Category in Computer Science}
\subsection{Motivation}
\subsection{Kleisli Triple}

\section{The Kleisli Category and Monads}
\subsection{Adjuntions}
\subsection{$\mathbb{T}$-algebras}
\subsection{The Kleisli Category}

\section{Monads in Haskell}
\subsection{Haskell}
\subsection{The Monad Typeclass}
\subsection{Haskell Monads are Monads}
\subsection{Kleisli triples are Haskell Monads}

\end{document}

