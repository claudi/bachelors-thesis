\RequirePackage[l2tabu]{nag}
\documentclass[a4paper]{article}
\usepackage[T1]{fontenc}
\usepackage[utf8]{inputenc}
\usepackage{amsmath}
\usepackage{amssymb}
\usepackage{amsthm}
\usepackage{lmodern}
\usepackage{microtype}
\usepackage{tikz}
\usetikzlibrary{cd}
\usepackage{biblatex}
\bibliography{refs.bib}
\usepackage{minted}
\usepackage[hidelinks, pdfencoding=auto]{hyperref}

\theoremstyle{plain}
\newtheorem{theorem}{Theorem}[section]
\newtheorem{proposition}[theorem]{Proposition}
\newtheorem{corollary}[theorem]{Corollary}
\newtheorem{example}[theorem]{Example}

\theoremstyle{definition}
\newtheorem{definition}[theorem]{Definition}
\newtheorem{notation}[theorem]{Notation}

\DeclareMathOperator{\Obj}{Obj}
\DeclareMathOperator{\Hom}{Hom}

\newcommand{\id}{\mathrm{id}}
\newcommand{\Set}{\mathrm{Set}}
\newcommand{\Hask}{\ensuremath{\mathrm{Hask}}}
\newcommand{\ev}{\mathrm{ev}}

\newcommand{\cat}[1]{\mathcal{#1}}

\newcommand{\ie}{{i}.{e}., }
\newcommand{\eg}{{e}.{g}. }

\title{Monads in Haskell}
\author{Claudi Lleyda Moltó}
\date{}

\begin{document}
\maketitle
\begin{abstract}
    Upon observing that applying a certain restriction to functions in
    programming, we obtain some substantial benefits but also some important
    apparent limitations, we introduce an abstraction through category theory
    that seems to solve all of the deficits, leaving us in a ``all of the good,
    none of the bad'' situation. We also see how this abstraction can be
    successfully implemented in a real-life programming language, namely
    Haskell.
\end{abstract}
\tableofcontents
\clearpage

\section{Motivation}
\label{sec:motivation}
We begin with an observation: functions found in computer programming may not
behave like functions in mathematics.

Unfortunately, it is common to call computer functions and mathematical
functions both functions. In this report we will do so, as it is usually clear
by the context, and only specifying when there might be confusion. In these
situations, we may refer to a computer function as a~\emph{procedure}.

An example of this is the \mintinline{C}{malloc()} function found in
the~\mintinline{C}{stdlib.h} library in the~\mintinline{C}{C} programming
language, which takes a size in number of bits as input and outputs an unique
pointer to an address to a block of memory allocated to fit the requested size,
which is marked as reserved. This function

\begin{enumerate}
    \item generally outputs a different pointer on calls with the same input,
    \item modifies global state by marking allocated memory as reserved.
\end{enumerate}

These properties are incompatible with the definition of a mathematical
function. The first one means it is not well defined, and the second one means
it has side effects, which mathematical functions do not have. This motivates
the following definitions:

\begin{definition}[Pure function]
    We say that a procedure that
    \begin{enumerate}
        \item outputs the same result for the same inputs (\ie is well defined),
        \item does not modify global state.
    \end{enumerate}
    is~\emph{pure}.
    Conversely, we say a procedure that is not pure is~\emph{impure}.
\end{definition}

So, the~\mintinline{C}{malloc()} function is impure, and a function that outputs
the result of squaring an integer is pure.

In theory, a compiler could use its knowledge of certain functions being pure to
better reason about code, which can be used to obtain benefits such as

\begin{enumerate}
    \item If a pure function is called more than once with the
        same input, the result calculated on the previous run can be reused
        without having to call the function again~\cite{frostMemorization}. This
        is called~\emph{memorization}

    \item Since pure functions have no side-effects, it is usually safe to call
        them in parallel \cite{SussPureFunctionParallelization}, leading to
        great optimization.
\end{enumerate}

While pure functions present some significant benefits, they also come with some
apparent deficits. For example, it is not obvious how pure functions could react
to user input or output anything to the screen, such as debugging information or
logs. They also seem to be limited to not using global variables.

Most languages circumvent these problems by providing an interface to define
both pure and impure functions. In this way, some of the benefits can be
applied, and the deficits can be handled by impure functions.

As of this point, the problems we mentioned are only apparent, and while it is
not at all obvious how they could be avoided, there is also no definitive proof
they cannot. We promise to provide an abstraction that solves these limitations,
and then we will see how it can be successfully implemented into a useful
programming language.

\section{Monads in Category Theory}
\subsection{Introduction to Category Theory}

\begin{definition}[Category]
    \label{def:category}
    A \emph{category}~\(\cat{C}\) is composed of a collection~\(\Obj(\cat{C})\)
    of \emph{objects}, for every two objects~\(A,B\in\Obj(\cat{C})\) a
    set~\(\Hom_{\cat{C}}(A,B)\) of \emph{morphisms} and a \emph{composition
    function}
    \[
        \circ:\Hom_{\cat{C}}(A,B)\times\Hom_{\cat{C}}(B,C) \longrightarrow
        \Hom_{\cat{C}}(A,C)
    \]
    that satisfies
    \begin{enumerate}
        \item \emph{Associativity}: for every three
            morphisms~\(f\in\Hom_{\cat{C}}(A,B)\),~\(g\in\Hom_{\cat{C}}(B,C)\)
            and~\(h\in\Hom_{\cat{C}}(C,D)\)
            we have
            \[
                h \circ (g \circ f) = (h \circ g) \circ f.
            \]
        \item \emph{Left and right units}: for every
            object~\(A\in\Obj(\cat{C})\) there exists a unique
            element~\(\id_{A}\in\Hom_{\cat{C}}(A,A)\)
            such that for every morphism~\(f\in\Hom_{\cat{C}}(A,B)\) we have
            \[
                f \circ \id_{B} = f,
            \]
            and for every morphism~\(g\in\Hom_{\cat{C}}(B,A)\) we have
            \[
                \id_{A} \circ g = g.
            \]
    \end{enumerate}
\end{definition}

\begin{notation}
    We write~\(f:A\longrightarrow B\) or~\(A\overset{f}{\longrightarrow}B\) to
    mean~\(f\in\Hom_{\cat{C}}(A,B)\) and~\(gf\) to mean~\(g\circ f\).
\end{notation}

\begin{example}[Category of sets]
    \label{cat:set}
    There is a category~\(\Set\), where the objects are sets and morphisms are
    maps between sets.
\end{example}

\begin{definition}[Functor]
    \label{def:functor}
    A~\emph{functor}~\(T\) from a category~\(\cat{C}\) to a category~\(\cat{D}\)
    is a map~\(T:\Obj(\cat{C})\longrightarrow\Obj(\cat{D})\), and for every
    pair of objects~\(A,B\in\Obj{C}\) a
    map~\(T:\Hom_{\cat{C}}(A,B)\longrightarrow\Hom_{\cat{D}}(TA,TB)\),
    such that~\(T\) preserves
    \begin{enumerate}
        \item \emph{Composition}: for every two
            morphisms~\(f:A\longrightarrow B\),~\(g:B\longrightarrow C\)
            in~\(\cat{C}\) we have
            \[
                T(g \circ f) = T(g) \circ T(f).
            \]
        \item \emph{Units}: for each object~\(A\in\Obj{C}\) we have
            \[
                T(\id_{A}) = \id_{T(A)}.
            \]
    \end{enumerate}
\end{definition}

\begin{notation}
    We write~\(T:\cat{C}\longrightarrow\cat{D}\) to mean~\(T\) is a functor from
    a category~\(\cat{C}\) to a category~\(\cat{D}\).
    We may also shorten~\(T(A)\) to~\(TA\) or~\(T(f)\) to~\(Tf\), given~\(A\) an
    object in~\(\cat{C}\) and~\(f\) a morphism in~\(\cat{C}\).
\end{notation}

\begin{definition}[Natural transformation]
    \label{def:natural-transformation}
    Let~\(S,T:\cat{C}\longrightarrow\cat{D}\) be functors. A \emph{natural
    transformation}~\(\tau:S\longrightarrow T\) is a
    family~\(\tau=\{\tau_{A}:SA\longrightarrow TA\}_{A\in\Obj(\cat{C})}\) of
    morphisms in~\(\cat{D}\) such that for every
    morphism~\(f:A\longrightarrow B\) in~\(\cat{C}\)
    \[
        Tf \circ \tau_{A} = \tau_{B} \circ Sf,
    \]
    that is, making the following diagram commutative
    \[
        \begin{tikzcd}
            SA \arrow[r, "\tau_{A}"] \arrow[d, "Sf"] & TA \arrow[d, "Tf"] \\
            SB \arrow[r, "\tau_{B}"] & TB
        \end{tikzcd}
    \]
\end{definition}

\subsection{Monads}
\begin{definition}[Monad]
    \label{def:monad}
    A~\emph{monad} over a category~\(\cat{C}\) is a triple~\((T,\eta,\mu)\)
    where~\(T:\cat{C}\longrightarrow\cat{C}\) is a functor
    and~\(\eta:\id\longrightarrow T\) and~\(\mu:TT\longrightarrow T\) are two
    natural transformations such that for every object~\(A\in\Obj(\cat{C})\) the
    diagrams
    \[
        \begin{tikzcd}
            TA \arrow[r, "T\eta_{A}"] & TTA \arrow[d, "\mu_{A}"] & TA \arrow[l,
            swap, "\eta_{TA}"] \\
                                      & TA \arrow[ul, equal] \arrow[ur, equal] &
        \end{tikzcd}
        \qquad\text{and}\qquad
        \begin{tikzcd}
            TTTA \arrow[r, "\mu_{TA}"] \arrow[d, "T\mu_{A}"] & TTA \arrow[d,
            "\mu_{A}"] \\
            TTA \arrow[r, "\mu_{A}"] & TA
        \end{tikzcd}
    \]
    are commutative. This is
    \[
        \mu_{A}\circ T\eta_{A} = \id_{TA} = \mu_{A}\circ\eta_{TA}
        \qquad\text{and}\qquad
        \mu_{A}\circ \mu_{TA}
        = \mu_{A} \circ T\mu_{A}.
    \]
    We call~\(\eta\) and~\(\mu\) the \emph{unit} and the \emph{multiplication}
    of the monad, respectively.
\end{definition}

\begin{example}[Pointed set monad]
    \label{monad:maybe}
    We define the~\emph{pointed set monad}~\((M,\eta,\mu)\) on the category of
    sets as follows.
    \begin{enumerate}
        \item The functor~\(M\) for any object~\(A\) or
            morphism~\(f:A\longrightarrow B\) is defined as
            \begin{gather*}
                MA = A\sqcup\{\bot\}
                \qquad\text{and}\qquad
                \begin{split}
                    Mf:MA&\longrightarrow MB \\
                    \bot&\longmapsto\bot \\
                    A\ni x&\longmapsto f(x)
                \end{split}
            \end{gather*}
        \item The unit~\(\eta\) for any object~\(A\) is defined as the inclusion
            of~\(A\) into~\(MA\).
        \item The multiplication~\(\mu\) for any object~\(A\) is defined as
            \begin{align*}
                \mu_{A}:MMA&\longrightarrow MA \\
                \bot_{1}&\longmapsto \bot \\
                \bot_{2}&\longmapsto \bot \\
                A\ni x&\longmapsto x
            \end{align*}
            where we have denoted~\(MMA=A\sqcup\{\bot_{1}\}\sqcup\{\bot_{2}\}\)
            and~\(MA=A\sqcup\{\bot\}\) for convenience.
    \end{enumerate}
\end{example}

\begin{example}[Power set monad]
    \label{monad:power-set}
    We define the~\emph{power set monad}~\((P,\eta,\mu)\) on the category of
    sets as follows.
    \begin{enumerate}
        \item The functor~\(P\) for any object~\(A\) or
            morphism~\(f:A\longrightarrow B\) is defined as
            \begin{gather*}
                PA = \mathcal{P}(A)
                \qquad\text{and}\qquad
                \begin{split}
                    Pf:PA&\longrightarrow PB \\
                    S&\longmapsto\{f(x)\in B \mid x\in S\}
                \end{split}
            \end{gather*}
        \item The unit~\(\eta\) for any object~\(A\) is defined
            as~\(\eta_{A}(x)=\{x\}\).
        \item The multiplication~\(\mu\) for any object~\(A\) is defined as
            \begin{align*}
                \mu_{A}:PPA&\longrightarrow PA \\
                S&\longmapsto \bigcup_{X\in S}X
            \end{align*}
    \end{enumerate}
\end{example}

\begin{definition}[Cartesian closed category]
    \label{def:product}
    \label{def:cartesian-closed}
    \label{def:exponential object}
    We say that a category~\(\cat{C}\) is~\emph{Cartesian closed} if
    \begin{enumerate}
        \item For any finite, possibly empty, family of
            objects~\(\{A_{i}\}_{i\in I}\) in~\(\cat{C}\) there is an
            object~\(A=\prod_{i\in I}A_{i}\) in~\(\cat{C}\).
        \item For any two objects~\(A\) and~\(B\) in~\(\cat{C}\) there is
            object~\(B^{A}\) together with a
            morphism~\(\ev:B^{A}\times A\longrightarrow B\) such that for every
            other object~\(C\) and morphism~\(g:C\times A\longrightarrow B\)
            in~\(\cat{C}\) there is a unique
            morphism~\(\lambda g:C\longrightarrow B^{A}\) such that the diagram
            \[\begin{tikzcd}
                C\times A \arrow[dr, "g"] \arrow[d, "\lambda g\times\id_{A}", swap] & \\
                B^{A}\times A \arrow[r, "\ev"] & B
            \end{tikzcd}\]
            commutes.
    \end{enumerate}
    We say that~\(B^{A}\) is the~\emph{exponential object}, and~\(\times\) is
    the~\emph{product}.
\end{definition}

\begin{example}
    The category of sets is Cartesian closed.
\end{example}

\section{The Kleisli Triples in Computer Science}
\label{sec:kleisli-triple}
In section~\ref{sec:motivation} we established that pure functions present some
deficits. Let's put category theory aside for a moment and see how we could work
around some of these, which we will then generalize through categorical notions.
\begin{example}[Non-determinism]
    \label{ex:kleisli-non-determinism}
    Suppose we have a procedure~\(A\longrightarrow B\) that has no side-effects
    but is non-deterministic.

    To model its behaviour through a mathematical function we could consider the
    function
    \[
        f:A\longrightarrow\mathcal{P}(B).
    \]
    that outputs all of the possible outputs of said procedure for a single
    output. Since the original procedure had no side-effects, and the new
    function is clearly deterministic, we can assure that~\(f\) is pure, and we
    can reason about it as a mathematical function.

    Given two such functions~\(f:A\longrightarrow\mathcal{P}(B)\)
    and~\(g:B\longrightarrow\mathcal{P}(C)\), we could compose them by
    extending~\(g\) to
    \begin{align*}
        g^{\ast}:\mathcal{P}(B)&\longrightarrow\mathcal{P}(C) \\
        S&\longmapsto\{g(x)\in C \mid x\in S\}
    \end{align*}
    and defining the composition of~\(f\) and~\(g\) to be~\(g^{\ast} \circ f\).
    With the same functions, notice that the composition and~\((-)^{\ast}\)
    satisfy
    \[
        g^{\ast}\circ f^{\ast}
        = (g^{\ast} \circ f)^{\ast}.
    \]

    As a last observation, we see how if we consider the inclusion
    \begin{align*}
        \eta_{A}:A&\longrightarrow\mathcal{P}(A) \\
        a&\longmapsto\{a\}
    \end{align*}
    we get that~\(\eta_{A}^{\ast}=\id_{\mathcal{P}(A)}\) and, with our new
    composition, we get~\(f^{\ast}\circ\eta_{A}=f\) for every
    function~\(f:A\longrightarrow\mathcal{P}(B)\).
\end{example}

\begin{example}[Side-effects]
    \label{ex:kleisli-side-effects}
    Suppose we have a procedure~\(A\longrightarrow B\) that is deterministic but
    presents side-effects.

    Modelling its behaviour through a mathematical function requires a bit more
    effort. A function that has side-effects is one that modifies global state
    beyond its scope. To make it pure it would be enough to also track the
    changes made to global state during its execution. We could achieve this by
    considering the function
    \[
        f':A\times S\longrightarrow B\times S
    \]
    where~\(S\) contains all possible global states, and~\(f'\) now outputs on
    the first component the result of the original procedure, while on the
    second it outputs the effected state.

    Notice how this also models the behaviour of a function that modifies its
    output depending on global state, even if it does not modify it, as the
    global state is part of its input.

    We can also consider the induced pure function
    \begin{align*}
        f:A&\longrightarrow(B\times S)^{S} \\
        a&\longmapsto\bigr(s\mapsto f'(a,s)\bigl)
    \end{align*}
    where~\(B^{A}\) is the exponential object, of which we are implicitly
    assuming existence.

    For convenience we define
    \[
        TA = (A\times S)^{S}.
    \]

    Now, given two functions~\(f:A\longrightarrow TB\)
    and~\(g:B\longrightarrow TC\), we could compose them by, again,
    extending~\(g\) to
    \begin{align*}
        g^{\ast}:TA&\longrightarrow TB \\
        s\mapsto\bigl(g_{1}(s),g_{2}(s)\bigr)&\longmapsto
        s\mapsto\bigl(g\circ g_{1}(s),g_{2}(s)\bigr)
    \end{align*}
    and defining their composition to be~\(g^{\ast} \circ f\).

    We can also define the inclusion
    \begin{align*}
        \eta_{A}:A&\longrightarrow TA \\
        a&\longmapsto\bigl(s\mapsto(a,s)\bigr)
    \end{align*}
    which satisfies~\(\eta_{A}^{\ast}=\id_{TA}\) and, for every
    function~\(f:A\longrightarrow TB\), we see it also
    satisfies~\(f^{\ast}\circ\eta_{A}=f\).
\end{example}

While not being problems inherent to pure functions, there are other situations
commonly found in computer programming that are also foreign to mathematical
functions, which pure functions behave like.

For example, it is difficult to specify the domain and codomain of a procedure,
which often leads to functions that are not defined on all of their input. An
instance of this is the factorial function, which may be written as
\begin{minted}{python}
def factorial(n: int):
    if n == 0:
        return 1
    elif n >= 1:
        return n * factorial(n - 1)
\end{minted}
This implementation of the factorial function accepts an integer as its input,
but it is not defined for negative values. We call this~\emph{partiality}.

\begin{example}[Partiality]
    \label{ex:kleisli-partiality}
    Suppose we have a procedure~\(A\longrightarrow B\) that has no side-effects
    and is deterministic, but instead is partial.

    To model its behaviour we can approach the problem by adding an ``special
    element'', denoted with~\(\bot\) to output in case we hit an input that does
    not have a defined output. This induces a function
    \[
        f:A\longrightarrow B\sqcup\{\bot\}
    \]
    Now, given two such functions~\(f:A\longrightarrow B\sqcup\{\bot\}\)
    and~\(g:B\longrightarrow C\sqcup\{\bot\}\) we can compose them by
    extending~\(g\) to~\(g^{\ast}:B\sqcup\{\bot\}\longrightarrow C\sqcup\{\bot\}\) as
    \begin{align*}
        g^{\ast}:B\sqcup\{\bot\}&\longrightarrow C\sqcup\{\bot\} \\
        b\in B&\longmapsto g(b) \\
        \bot&\longmapsto\bot
    \end{align*}
    and then defining their composition as~\(g^{\ast}\circ f\). With the same
    functions, notice that the composition and~\((-)^{\ast}\) satisfy
    \[
        g^{\ast}\circ f^{\ast} = (g^{\ast}\circ f)^{\ast}.
    \]

    We can also consider the inclusion
    \begin{align*}
        \eta_{A}:A&\longrightarrow A\sqcup\{\bot\} \\
        b&\longmapsto b
    \end{align*}
    and we get that~\(\eta_{A}^{\ast}=\id_{A\sqcup\{\bot\}}\) and, with our previously
    defined composition, we get~\(f^{\ast}\circ\eta_{A}=f\) for every
    function~\(f:A\longrightarrow B\sqcup\{\bot\}\).
\end{example}

Another example are~\emph{exceptions}. Procedures often encounter errors during
their execution, which mathematical functions do not, and must signal them to
other functions that have to deal with them, either by passing the exception up
the call chain (eventually aborting the program) or by directly dealing with it
during their execution.

\begin{example}[Exceptions]
    Suppose we have a procedure~\(A\longrightarrow B\) that has no side-effects
    and is deterministic, but might encounter an error and not be able to
    produce a meaningful result.

    In these situations it is common for functions to return a special value,
    which is not ideal if any output might be valid. More advanced programming
    languages have a system that can manage these errors by separating them from
    normal output. To model the behaviour of these functions, we could consider
    a function
    \[
        f:A\longrightarrow B\sqcup E
    \]
    which, for any execution of the procedure, if the computation is successful
    outputs its result in~\(B\), and if the procedure encounters an error, it
    outputs the error, which would fall in~\(E\).

    The set~\(E\) would be the set error identifiers, such as error codes, and
    we can interpret this induced function as adding information about the
    possible errors encountered to the output of the procedure. The considered
    function~\(f\) is now pure, and we can reason about it mathematically.

    We are once again interested in composing two such
    functions~\(f:A\longrightarrow B\sqcup E\)
    and~\(g:B\longrightarrow C\sqcup E\). For this we could extend~\(g\)
    to~\(g^{\ast}:B\sqcup E\longrightarrow C\sqcup E\) as
    \begin{align*}
        g^{\ast}:B\sqcup E&\longrightarrow C\sqcup E \\
        b\in B&\longmapsto g(b) \\
        e\in E&\longmapsto e
    \end{align*}
    and then defining their composition as~\(g^{\ast}\circ f\). Notice how, with
    the same functions, we can see that
    \[
        g^{\ast} \circ f^{\ast} = (f^{\ast} \circ g)^{\ast}.
    \]

    We can also consider the inclusion
    \begin{align*}
        \eta_{A}:A&\longrightarrow A\sqcup E \\
        a&\longmapsto a
    \end{align*}
    and we get that~\(\eta_{A}^{\ast}=\id_{A\sqcup E}\) and, with our previously
    defined composition, we get~\(f^{\ast}\circ\eta_{A}=f\) for every
    function~\(f:A\longrightarrow B\sqcup E\).
\end{example}

\subsection{Kleisli Triple}
\begin{definition}[Kleisli Triple]
    \label{def:kleisli-triple}
    We define a \emph{Kleisli Triple} over a category~\(\cat{C}\)
    as a triple~\((T, \eta, (-)^{\ast})\) consisting of
    \begin{enumerate}
        \item a class function
            \(T:\Obj(\cat{C})\longrightarrow\Obj(\cat{C})\).
        \item \(\forall A\in\cat{C}\)
            a morphism~\(\eta_{A}:A\longrightarrow TA\)
            in~\(\cat{C}\).
        \item \(\forall f:A\longrightarrow TB\)
            a morphism~\(f^{\ast}:TA\longrightarrow TB\).
    \end{enumerate}
    such that
    \begin{enumerate}
        \item For all~\(A\in\cat{C}\) we have
            \({\eta_{A}}^{\ast} = \id_{TA}:TA\longrightarrow TA\).
        \item For all~\(A,B\in\cat{C}\)
            and \(f:A\longrightarrow TB\)
            the diagram
            \[\begin{tikzcd}
                TA \arrow[r, "f^{\ast}"] & TB \\
                A \arrow[u, "\eta_{A}"] \arrow[ur, "f", swap] &
            \end{tikzcd}\]
            commutes. That is~\(f^{\ast}\circ\eta_{A} = f\).
        \item For all~\(f:A\to TB,g:B\to TC\) we
            have~\(g^{\ast}\circ f^{\ast} = (g^{\ast}\circ f)^{\ast}\).
    \end{enumerate}
\end{definition}

\begin{theorem}
    \label{thm:kleisli-triples-and-monads-correspondence}
    Let~\(\cat{C}\) be a category.
    There is a bijective correspondence between Kleisli Triples over~\(\cat{C}\)
    and monads over~\(\cat{C}\).
\end{theorem}
\begin{proof}
    Let~\((T, \eta, (-)^{\ast})\) be a Kleisli triple over~\(\cat{C}\) and set
    \begin{equation}
        \label{eq:monad-unit-in-kleisli-trip}
        \begin{split}
            T:\cat{C} & \longrightarrow\cat{C} \\
            A & \longmapsto TA \\
            f:A\rightarrow B & \longmapsto
            (\eta_{B}\circ f)^{\ast}:TA\rightarrow TB
        \end{split}
    \end{equation}
    and
    \begin{equation}
        \label{eq:monad-prod-in-kleisli-trip}
        \mu_{A} = (\id_{TA})^{\ast}.
    \end{equation}
    We will see that~\((T,\eta,\mu)\) is a monad.

    Let's check that~\(T\) is a functor.
    For every two~\(f:A\longrightarrow B\)
    and~\(g:B\longrightarrow C\)
    morphisms in~\(\cat{C}\)
    we have
    \begin{align*}
        Tg\circ Tf &= (\eta_{C}\circ g)^{\ast}\circ
                      (\eta_{B}\circ f)^{\ast} \\
                   &= \bigl((\eta_{C}\circ g)^{\ast}\circ
                      (\eta_{B}\circ f)\bigr)^{\ast} \\
                   &= \bigl((\eta_{C}\circ g)^{\ast}\circ
                      \eta_{B}\circ f\bigr)^{\ast} \\
                   &= (\eta_{C}\circ g\circ f)^{\ast} \\
                   &= T(g\circ f)
    \end{align*}
    and for all~\(A\in\cat{C}\),
    \[
        T\id_{A} = (\eta_{A}\circ \id_{A})^{\ast} = \eta_{A}^{\ast} = \id_{TA},
    \]
    which proves that~\(T\) is a functor.

    Next we want to see that~\(\mu=\{\mu_{A}\}_{A\in\Obj(\cat{C})}\) is a
    natural transformation.
    Given morphism~\(f:A\longrightarrow B\) in~\(\cat{C}\) we want to prove that
    \[
        Tf \circ \mu_{A} = \mu_{B} \circ TTf.
    \]
    Using~\eqref{eq:monad-prod-in-kleisli-trip} we get
    \begin{gather*}
        \begin{split}
            Tf \circ \mu_{A} &= Tf \circ (\id_{TA})^{\ast} \\
                &= (\eta_{B} \circ f)^{\ast} \circ (\id_{TA})^{\ast} \\
                &= \bigl((\eta_{B} \circ f)^{\ast} \circ \id_{TA}\bigr)^{\ast} \\
                &= \bigl((\eta_{B} \circ f)^{\ast}\bigr)^{\ast} \\
                &= (Tf)^{\ast}
        \end{split}
        \qquad\text{and}\qquad
        \begin{split}
            \mu_{B} \circ TTf &= (\id_{TB})^{\ast} \circ TTf \\
                &= (\id_{TB})^{\ast} \circ (\eta_{TB} \circ Tf)^{\ast} \\
                &= \bigl((\id_{TB})^{\ast} \circ \eta_{TB} \circ Tf\bigr)^{\ast} \\
                &= \bigl(\id_{TB} \circ Tf\bigr)^{\ast} \\
                %&= \bigl(T\id_{B} \circ Tf\bigr)^{\ast} \\
                &= (Tf)^{\ast}
        \end{split}
    \end{gather*}
    and we have proved that~\(\mu\) is a natural transformation.

    We can check that~\(\eta=\{\eta_{A}\}_{A\in\Obj(\cat{C})}\) is also a
    natural transformation.
    By using~\eqref{eq:monad-unit-in-kleisli-trip} we can see that, for any
    morphism~\(f:A\longrightarrow B\) in~\(\cat{C}\), we have
    \[
        Tf \circ \eta_{A}
        = (\eta_{B}\circ f)^{\ast} \circ \eta_{A}
        = \eta_{B} \circ f,
    \]
    which shows that~\(\mu\) is a natural transformation.

    Finally, we can check that~\(\mu\) and~\(\eta\) satisfy the monad laws
    \[
        \mu_{A}\circ T\eta_{A} = \id_{TA} = \mu_{A}\circ\eta_{TA}
        \qquad\text{and}\qquad
        \mu_{A}\circ \mu_{TA}
        = \mu_{A} \circ T\mu_{A}.
    \]
    Using the Kleisli triple laws calculate
    \begin{align*}
        \mu_{A} \circ T\eta_{A}
            &= \mu_{A} \circ (\eta_{TA} \circ \eta_{A})^{\ast} \\
            &= {\id_{TA}}^{\ast} \circ (\eta_{TA} \circ \eta_{A})^{\ast} \\
            &= ({\id_{TA}}^{\ast} \circ \eta_{TA} \circ \eta_{A})^{\ast} \\
            &= \bigl(({T\id_{A}}^{\ast} \circ \eta_{TA}) \circ \eta_{A}\bigr)^{\ast} \\
            &= (T\id_{A} \circ \eta_{A})^{\ast} \\
            &= {\eta_{A}}^{\ast} \\
            &= \id_{TA} \\
            &= {\id_{TA}}^{\ast} \circ \eta_{TA} \\
            &= \mu_{A} \circ \eta_{TA}
    \end{align*}
    and
    \begin{align*}
        \mu_{A} \circ \mu_{TA}
            &= {\id_{TA}}^{\ast} \circ {\id_{TTA}}^{\ast} \\
            &= ({\id_{TA}}^{\ast} \circ \id_{TTA})^{\ast} \\
            &= {\id_{TA}}^{\ast\ast} \\
            &= \bigl(\id_{TA} \circ (\id_{TA})^{\ast}\bigr)^{\ast} \\
            &= \bigl({\id_{TA}}^{\ast} \circ \eta_{TA}\circ(\id_{TA})^{\ast}\bigr)^{\ast} \\
            &= {\id_{TA}}^{\ast} \circ \bigl(\eta_{TA}\circ(\id_{TA})^{\ast}\bigr)^{\ast} \\
            &= {\id_{TA}}^{\ast} \circ T(\id_{TA})^{\ast} \\
            &= \mu_{A} \circ T\mu_{A}
    \end{align*}
    and this proves that a Kleisli Triple induces a monad.

    Let~\((T,\eta,\mu)\) be a monad over~\(\cat{C}\) and for any
    morphism~\(f:A\longrightarrow TB\) in~\(\cat{C}\) set
    \begin{equation}
        \label{eq:kleisli-ast-in-monad}
        f^{\ast} = \mu_{B} \circ Tf.
    \end{equation}
    We want to check that~\((T,\eta,(-)^{\ast})\) is a Kleisli triple.

    By the monad unit law we have
    \[
        {\eta_{TA}}^{\ast} = \mu_{TA} \circ T\eta_{TA} = \id_{TA},
    \]
    and for any two morphisms~\(f:A\longrightarrow TB\)
    and~\(g:B\longrightarrow TC\) in~\(\cat{C}\) we have
    \begin{align*}
        g^{\ast}\circ f^{\ast} &= \mu_{C} \circ Tg \circ \mu_{B} \circ Tf \\
                               &= \mu_{C} \circ \mu_{TC} \circ TTg \circ Tf \\
                               &= \mu_{C} \circ T\mu_{C} \circ TTg \circ Tf \\
                               &= \mu_{C} \circ T(\mu_{C} \circ Tg \circ f) \\
                               &= \mu_{C} \circ T(g^{\ast} \circ f) \\
                               &= (g^{\ast} \circ f)^{\ast}
    \end{align*}

    Then, for any morphism~\(f:A\longrightarrow TB\) in~\(\cat{C}\) we can use
    the unit law to get
    \begin{align*}
        f^{\ast} \circ \eta_{A}
            &= \mu_{B} \circ Tf \circ \eta_{A} \\
            &= \mu_{B} \circ \eta_{TB} \circ f \\
            &= \id_{TB} \circ f \\
            &= f
    \end{align*}
    and this proves that every monad induces a Kleisli Triple.

    It is clear that these two constructions are inverse to each other.
\end{proof}

\subsection{The Kleisli category}
For this definition to be useful we must also show that the defined morphisms
are well behaved in some sense. At the very minimum, we should be able to
compose them and there should be a unit function.

We show that they actually form a category, and give an explicit formula for
their composition and units.

\begin{proposition}[The Kleisli category]
    Given a Kleisli triple~\((T,\eta,(-)^{\ast})\) over a category~\(\cat{C}\),
    the following data forms a category~\(\cat{C}_{T}\), which we call
    the~\emph{Kleisli category}.
    \begin{enumerate}
        \item The objects~\(\Obj(\cat{C}_{T}) = \Obj(\cat{C})\).
        \item For any two objects~\(A,B\) in~\(\cat{C}_{T}\), the morphisms
            \[
                \Hom_{\cat{C}_{T}}(A,B) = \Hom_{C}(A,TB).
            \]
        \item For any two
            morphisms~\(f:A\longrightarrow TB\),~\(g:B\longrightarrow TC\)
            in~\(\cat{C}_{T}\), the composition
            \[
                g \circ_{T} f = g^{\ast} \circ f.
            \]
    \end{enumerate}
\end{proposition}
\begin{proof}
    We start by checking that the composition is associative. For any three
    morphisms~\(f:A\longrightarrow TB\),~\(g:B\longrightarrow TC\)
    and~\(h:C\longrightarrow TD\) in~\(\cat{C}_{T}\) we have
    \begin{align*}
        h \circ_{T} (g \circ_{T} f)
            &= h^{\ast} \circ (g \circ_{T} f) \\
            &= h^{\ast} \circ (g^{\ast} \circ f) \\
            &= (h^{\ast} \circ g^{\ast}) \circ f \\
            &= (h^{\ast} \circ g)^{\ast} \circ f \\
            &= (h^{\ast} \circ g) \circ_{T} f \\
            &= (h \circ_{T} g) \circ_{T} f,
    \end{align*}
    were we have used the third property of Kleisli triples.

    We can also see that for every object~\(A\) in~\(\Obj(\cat{C}_{T})\) there
    is a unit morphism~\(\id_{A}\). If we take
    \[
        \id_{A} = \eta_{A}
    \]
    then for every morphism~\(f:A\longrightarrow TB\) in~\(\cat{C}_{T}\) we have
    \begin{align*}
        f \circ_{T} \id_{A}
        &= f \circ_{T} \eta_{A} \\
        &= f^{\ast} \circ \eta_{A} \\
        &= f
    \end{align*}
    and for every morphism~\(g:B\longrightarrow TA\) in~\(\cat{C}_{T}\) we have
    \begin{align*}
        \id_{A} \circ_{T} g
        &= \eta_{A} \circ_{T} g \\
        &= \eta_{A}^{\ast} \circ g \\
        &= \id_{TA} \circ g \\
        &= g
    \end{align*}
    by using the first and second properties of Kleisli triples.

    This is enough to check that the construction satisfies the definition of a
    category given in~\ref{def:category}.
\end{proof}

\section{Monads in Haskell}
So far we have described functional programming at a theoretical level. We
emphasized its apparent drawbacks and proposed an abstract solution for them
through the use of Kleisli triples in section~\ref{sec:kleisli-triple}, but we
are interested in how this solution can be implemented, since abstractions might
not always be useful or feasible in practice.

For this we use Haskell, a language that makes use of these abstract concepts to
be a programming language where programmers are only allowed to use pure
functions by default. Such languages are called~\emph{purely functional} or said
to belong to the~\emph{purely functional programming
paradigm}~\cite{paradigms-overview}.

\subsection{Haskell}
To achieve this, the Haskell language makes use of notation that is atypical
outside of such languages. In this section we aim to introduce the basics of it
so that we can then use it to explain how the theory we discussed previously is
applied. More thorough introductions can be found in~\cite{haskell-org-docs}.

\subsubsection{Types and functions}
In Haskell, any element belongs to a particular~\emph{type}. Examples of types
are~\mintinline{Haskell}{Int} and~\mintinline{Haskell}{Char}, used for integer
numbers and characters respectively. To denote an element belonging to a certain
type we write
\begin{minted}{Haskell}
x :: Int
\end{minted}
which reads
\begin{quote}
    The element \mintinline{Haskell}{x} belongs to the~\mintinline{Haskell}{Int}
    type.
\end{quote}
The~\mintinline{Haskell}{::} symbol can be seen as the~\(\in\) symbol in math,
and types may be regarded as sets for the most part.

There are also types of functions. For any two types~\mintinline{Haskell}{a}
and~\mintinline{Haskell}{b} there is a type
\begin{minted}{Haskell}
a -> b
\end{minted}
whose elements are functions from the first type to the second. This means that
functions in Haskell take only one argument, which is written after the function
name, without parenthesis and separated by spaces. For example, if we have
a~\mintinline{Haskell}{factorial :: Int -> Int} function and we want to evaluate
it at~\mintinline{Haskell}{5 :: Int} we write
\begin{minted}{Haskell}
factorial 5
\end{minted}

Concrete types like~\mintinline{Haskell}{Int} and~\mintinline{Haskell}{Char} are
written starting with an uppercase letter, while generic types are written in
lowercase.

Of course, since the function types are types, we can have functions that take
or output other functions. An example would be
\begin{minted}{Haskell}
f :: a -> (b -> c)
g :: (a -> b) -> c
\end{minted}
Here~\mintinline{Haskell}{f} is a function that takes an element of
type~\mintinline{Haskell}{a} and returns an element of
type~\mintinline{Haskell}{b -> c}, \ie a function from~\mintinline{Haskell}{b}
to~\mintinline{Haskell}{c}, while~\mintinline{Haskell}{g} is a function that
takes a function from~\mintinline{Haskell}{a} to~\mintinline{Haskell}{b} as an
argument an returns an element of type~\mintinline{Haskell}{c}. If we
regard~\mintinline{Haskell}{->} as an operator, we say it is right associative,
and we can write
\begin{minted}{Haskell}
f :: a -> b -> c
\end{minted}
instead of
\begin{minted}{Haskell}
f :: a -> (b -> c)
\end{minted}

We can use this simulate functions with multiple arguments.
If we consider a function~\mintinline{Haskell}{sum :: Int -> Int -> Int} that
adds two integers and two elements~\mintinline{Haskell}{41 :: Int}
and~\mintinline{Haskell}{19 :: Int} we would have
\begin{minted}{Haskell}
sum       :: Int -> Int -> Int -- 1
sum 41    :: Int -> Int        -- 2
sum 41 19 :: Int               -- 3
\end{minted}
where
\begin{enumerate}
    \item The~\mintinline{Haskell}{sum :: Int -> Int -> Int} function takes an
        argument of type~\mintinline{Haskell}{Int} and outputs a function of
        type~\mintinline{Haskell}{Int -> Int}.
    \item After evaluating~\mintinline{Haskell}{sum :: Int -> Int -> Int}
        on~\mintinline{Haskell}{41 :: Int} we get a
        function~\mintinline{Haskell}{Int -> Int},
        denoted~\mintinline{Haskell}{sum 41}.
        Intuitively,~\mintinline{Haskell}{sum 41} is a function that takes an
        integer and returns the result of adding~\(41\) to that integer.
    \item We can evaluate the function~\mintinline{Haskell}{sum 41}
        on~\mintinline{Haskell}{19 :: Int} to get a final result of
        type~\mintinline{Haskell}{Int}.
\end{enumerate}

We can also define operations, or infix functions. To do this we surround the
name of the function with parentheses as such
\begin{minted}{Haskell}
(+) :: Int -> Int -> Int
\end{minted}
and we can later write
\begin{minted}{Haskell}
1 + 2 :: Int
\end{minted}

\subsubsection{Function bodies}
To explain how to define the body of a function we start with an example
\begin{minted}{Haskell}
square :: Int -> Int -- 1
square n = n * n     -- 2
\end{minted}
here
\begin{enumerate}
    \item We first declare that we will define a
        function~\mintinline{Haskell}{square} of
        type~\mintinline{Haskell}{Int -> Int}.
    \item Here we define the behaviour of our function given an
        argument~\mintinline{Haskell}{n}. The notation is similar to the one
        we use in math, where we write \(f(n) = n*n\).
\end{enumerate}
Another example
\begin{minted}{Haskell}
multiply :: Int -> Int -> Int
multiply n m = n * m
\end{minted}
although functions in Haskell technically only take one argument, we can define
a function that takes multiple parameters in one line as shown in the example
above.

Definitions like
\begin{minted}{Haskell}
multiplyByTwo :: Int -> Int
multiplyByTwo = multiply 2
\end{minted}
are also valid. Here the function~\mintinline{Haskell}{multiplyByTwo} is defined
in terms of the function~\mintinline{Haskell}{multiply 2}.

\subsubsection{Algebraic data types}
We can also define new types from existing ones. We already know how to do this
with~\mintinline{Haskell}{->}, but we can also use
the~\mintinline{Haskell}{data} keyword
\begin{minted}{Haskell}
data Complex = Cartesian Double Double
\end{minted}
which consists of a space-separated blueprint~\mintinline{Haskell}{Cartesian
Double Double} for the new~\mintinline{Haskell}{Complex} type.

This defines a new called~\mintinline{Haskell}{Complex}. This definition comes
with an associated
map~\mintinline{Haskell}{Cartesian :: Double -> Double -> Complex} which lets us
instantiate elements in the~\mintinline{Haskell}{Cartesian} type, \ie
\begin{minted}{Haskell}
Cartesian 1.0 2.0 :: Complex
\end{minted}

Here we constructed a type~\mintinline{Haskell}{Complex} from
two~\mintinline{Haskell}{Double} types. We can see this as the Cartesian product
of two types, which in math we would
write~\(\mathbb{C}=\mathbb{R}\times\mathbb{R}\).

We can also define new types in a similar fashion to the union of sets with
\begin{minted}{Haskell}
data Shape = Circle Double | Polygon Int Double
\end{minted}
where~\mintinline{Haskell}{|} reads as ``or''. Here, a member of
the~\mintinline{Haskell}{Shape} type could be either
a~\mintinline{Haskell}{Circle}, specified by the length of
type~\mintinline{Haskell}{Double} of its radius or a
(regular)~\mintinline{Haskell}{Polygon}, specified the number of sides of
type~\mintinline{Haskell}{Int} and its apothem length, also of
type~\mintinline{Haskell}{Double}.

So far we are capable of defining new types by addition and multiplication. We
call the first kind of types~\emph{sum types} and the second type~\emph{product
types}. Together it is said that the types form an algebraic structure, and
types made combining these operations are~\emph{algebraic data types}.

The definition of~\mintinline{Haskell}{Shape} induces two functions
\begin{minted}{Haskell}
Circle  :: Double -> Shape
Polygon :: Int -> Double -> Shape
\end{minted}
but these words, in this case~\mintinline{Haskell}{Circle}
and~\mintinline{Haskell}{Polygon}, acquire another meaning when defying a new
type. To implement a function to calculate the area of our shapes we would write
\begin{minted}{Haskell}
area :: Shape -> Double
area (Circle r)    = pi*r*r
area (Polygon n a) = n*a*a*tan(pi/n)
\end{minted}
which introduces the other use of the words~\mintinline{Haskell}{Circle}
and~\mintinline{Haskell}{Polygon} we used to define~\mintinline{Haskell}{Shape}.
They are used for~\emph{pattern matching}. To evaluate
the~\mintinline{Haskell}{area} function on an element we must
first be able to identify what kind of~\mintinline{Haskell}{Shape} it is. This
is done through matching the instance of said element on the constructors, and
the program will make sure to execute the adequate one.

Pattern matching also applies to the previous example:
\begin{minted}{Haskell}
conjugate :: Complex -> Complex
conjugate (Cartesian x y) = Cartesian x (-y)
\end{minted}
notice here the use of~\mintinline{Haskell}{Cartesian} to pattern match and to
construct an element of the~\mintinline{Haskell}{Complex} type.

We do not always have to pattern match. We can avoid it when we can treat the
different options equally and we do not need to access the parameters used to
construct the instance of the type, for example
\begin{minted}{Haskell}
sameArea :: Shape -> Shape -> Bool
sameArea s1 s2 = (area s1) == (area s2)
\end{minted}

\subsubsection{Parametrized data types}
So far we have seen how we can define types from other concrete types. The
Haskell language also allows us to define types parametrized by other types, for
example we can define a binary tree type to hold nodes of any type as
\begin{minted}{Haskell}
data Pair a = Tuple a a
\end{minted}
Here~\mintinline{Haskell}{a} is a generic type, and acts as a parameter to the
type definition. We can then instantiate a pair
of~\mintinline{Haskell}{Int} elements as
\begin{minted}{Haskell}
Tuple 2 3 :: Pair Int
\end{minted}

Examples of generic functions over the type are
\begin{minted}{Haskell}
first :: Pair a -> a
first (Tuple x _) = x

second :: Pair a -> a
second (Tuple _ y) = y

flip :: Pair a -> Pair a
flip (Tuple x y) = Tuple y x

apply :: (a -> a -> b) -> Pair a -> b
apply f (Tuple x y) = f x y
\end{minted}
here we have used~\mintinline{Haskell}{_} for the unused parameters
in~\mintinline{Haskell}{first} and~\mintinline{Haskell}{second}. This is because
they are clearly not used in the function body, and if we assign a variable name
to an unused argument the Haskell compiler will throw an error.

Generic types can be parametrized by more than one type. We can generalize
the~\mintinline{Haskell}{Pair} into
\begin{minted}{Haskell}
data Pair a b = Tuple a b
\end{minted}
and then we would have
\begin{minted}{Haskell}
Tuple 3 'c' :: Pair Int Char

first :: Pair a b -> a
first (Tuple x _) = x

second :: Pair a b -> b
second (Tuple _ y) = y

flip :: Pair a b -> Pair b a
flip (Tuple x y) = Tuple y x

apply :: (a -> b -> c) -> Pair a b -> c
apply f (Tuple x y) = f x y
\end{minted}

\subsection{The \texorpdfstring{\mintinline{Haskell}{Monad}}{Monad} typeclass}
The Haskell language allows users to define~\emph{typeclasses}, which area set
of constraints we give to a certain set of types.
They are similar to interfaces, which appear in other languages.

For our purposes, we can picture them as a definition. A typeclass describes a
behavior we expect.

Let's see an example. The~\mintinline{Haskell}{Functor} typeclass is
\begin{minted}{Haskell}
class Functor f where
    fmap :: (a -> b) -> f a -> f b
\end{minted}
From the usage of~\mintinline{Haskell}{f}, the Haskell compiler can infer
that it must be a parametric type with one parameter, and
with this information we can read the previous statement as
\begin{quote}
    For a parametric datatype~\mintinline{Haskell}{f} with one parameter to be
    called a Functor, it must be equipped with a structure

    \mintinline{Haskell}{fmap :: (a -> b) -> f a -> f b}
\end{quote}

Let's see an example of an implementation. If we define the parametric data
type~\mintinline{Haskell}{Maybe} as follows
\begin{minted}{Haskell}
data Maybe a = Just a | None
\end{minted}
we can make it into a Functor by writing
\begin{minted}{Haskell}
instance Functor Maybe where
    fmap :: (a -> b) -> Maybe a -> Maybe b
    fmap _ Nothing  = Nothing
    fmap f (Just x) = Just (f x)
\end{minted}
where we first declare that we want to show that~\mintinline{Haskell}{Maybe a}
is a~\mintinline{Haskell}{Functor}, which we do by providing a definition
for~\mintinline{Haskell}{fmap :: (a -> b) -> Maybe a -> Maybe b}.

Given a parametric datatype~\mintinline{Haskell}{f a}, the functor typeclass
provides a function that can be understood as having
type~\mintinline{Haskell}{fmap :: (a -> b) -> (f a -> f b)}, which reminds us of
the property of functors to act on functions. Additionally, the parametric
datatype itself acts on types, by transforming an existing type into a new one,
behaviour similar to that of a functor.

This is not enough for a function on objects and morphisms to be a functor. The
functor laws must be satisfied as well. The Haskell language does not provide a
way to express such laws, since for this it would need a way to write and
validate proofs, which would add too much complexity to the language. The
compiler assumes any typeclass implementations it encounters satisfy such
properties, which are only generally found in external documentation. Failing to
satisfy these laws could lead to buggy code and unintended behaviour.

We will not prove typeclass laws for specific instances we provide in this
section, as the proofs are usually straightforward but uncomfortable to write.

The laws for the functor typeclass are
\begin{minted}{Haskell}
fmap id = id
fmap (f . g) = (fmap f . fmap g)
\end{minted}
that are clearly equivalent to the functor laws found in~\ref{def:functor}.
The~\mintinline{Haskell}{f} that appears in these laws is different from
the~\mintinline{Haskell}{f} found in the definition of
the~\mintinline{Haskell}{Functor} typeclass, since it now corresponds to an
arbitrary function. This misleading use of notation happens often in Haskell,
and will be repeated in further typeclass laws in this document.

There is a notion of~\emph{inheritance} of typeclasses. Typeclass inheritance is
when a typeclass has a superclass. This is a way of expressing that a typeclass
requires another typeclass to be available for a given type before you can write
an instance. For example, Haskell provides a class that inherits
from~\mintinline{Haskell}{Functor}
\begin{minted}{Haskell}
class Functor f => Applicative f where
    (<*>) :: f (a -> b) -> f a -> f b
    pure :: a -> f a
\end{minted}
the~\mintinline{Haskell}{=>} notation is unfortunate, as in math it is usually
associated with an implication.

In this situation, to prove that a datatype is
an~\mintinline{Haskell}{Applicative}, we must first prove it is
a~\mintinline{Haskell}{Functor}, and then provide implementations for the two
required functions.
The laws for the~\mintinline{Haskell}{Applicative} typeclass are
\begin{minted}{Haskell}
pure id <*> v = v
pure (.) <*> u <*> v <*> w = u <*> (v <*> w)
pure f <*> pure x = pure (f x)
u <*> pure y = pure ($ y) <*> u
\end{minted}
and we can make~\mintinline{Haskell}{Maybe}, which we know to be
a~\mintinline{Haskell}{Functor}, into an~\mintinline{Haskell}{Applicative}
\begin{minted}{Haskell}
instance Applicative Maybe where
    (<*>) :: Maybe (a -> b) -> Maybe a -> Maybe b
    (<*>) (Just f) m = fmap f m
    (<*>) _ _ = Nothing

    pure :: a -> Maybe a
    pure = Just
\end{minted}
this class coincides with the notion of lax monoidal functor, which is not
related to our topic. They are used to define the~\mintinline{Haskell}{Monad}
typeclass.

\begin{minted}{Haskell}
class Applicative m => Monad m where
    (>>=) :: m a -> (a -> m b) -> m b
    (>>) :: m a -> m b -> m b
    return :: a -> m a
\end{minted}
where~\mintinline{Haskell}{(>>=)} is called~\emph{bind}.
Any~\mintinline{Haskell}{Monad} implementation must satisfy
\begin{minted}{Haskell}
return a >>= k = k a
m >>= return = m
m >>= (\x -> k x >>= h) = (m >>= k) >>= h

k >> f = k >>= \_ -> f

pure = return
m1 <*> m2 = m1 >>= (x1 -> m2 >>= (x2 -> return (x1 x2)))
\end{minted}
finally, we can make~\mintinline{Haskell}{Maybe}, which we know to be
an~\mintinline{Haskell}{Applicative}, into a~\mintinline{Haskell}{Monad}
\begin{minted}{Haskell}
instance Monad Maybe where
    (>>=) :: Maybe a -> (a -> Maybe b) -> Maybe b
    Just x  >>= k = k x
    Nothing >>= _ = Nothing

    (>>) :: Maybe a -> Maybe b -> Maybe b
    Just _  >> m = m
    Nothing >> _ = Nothing

    return :: a -> Maybe a
    return = pure
\end{minted}

\subsection{Haskell monads are monads}
The~\mintinline{Haskell}{Monad} structure defined by Haskell
\begin{minted}{Haskell}
class Applicative m => Monad m where
    (>>=) :: m a -> (a -> m b) -> m b
    (>>) :: m a -> m b -> m b
    return :: a -> m a
\end{minted}
subject to the laws
\begin{minted}{Haskell}
return a >>= k = k a
m >>= return = m
m >>= (\x -> k x >>= h) = (m >>= k) >>= h

k >> f = k >>= \_ -> f

pure = return
m1 <*> m2 = m1 >>= (x1 -> m2 >>= (x2 -> return (x1 x2)))
\end{minted}
does not correspond, at first sight, to the definition of monad given
in~\ref{def:monad}.

We begin observing that the~\mintinline{Haskell}{Monad}
operation~\mintinline{Haskell}{(>>)} can be uniquely determined in terms
of~\mintinline{Haskell}{(>>=)}, the bind operation.
This means that we can omit this operation from our reasoning. In fact, it is
only defined because it is useful in a programming setting, and bears no
particular mathematical relevance to us.

Our next observation lies on the bind operation. If we flip its arguments, we
get an operation with type signature
\begin{minted}{Haskell}
(a -> m b) -> (m a -> m b)
\end{minted}
which coincides with the~\((-)^{\ast}\) operation, while the type signature for
the~\mintinline{Haskell}{return} function coincides with~\(\eta_{A}\)
(where~\(A\) is determined by context in the Haskell program), as defined in the
definition of Kleisli triple in~\ref{def:kleisli-triple}.

With this correspondence we can now prove that the~\mintinline{Haskell}{Monad}
structure defined by Haskell is a monad by first showing that it is a Kleisli
triple, and then using
theorem~\ref{thm:kleisli-triples-and-monads-correspondence}. This can be shown
directly from the first three~\mintinline{Haskell}{Monad} laws.

\section{The influence of Haskell}
The Haskell language can be regarded as a successful proof of concept on the
usefulness and viability of pure functions, and this has led most traditional
programming languages to adopt some of the features that research on functional
programming has brought to light. Modern programming languages are designed with
support for functional programming or pure functions~\cite{enwiki:1023837642},
and plenty of old programming languages now support the functional programming
paradigm.

For example, the~\mintinline{C}{C} programming language, a famously imperative
and state-full language supports a~\mintinline{C}{pure} attribute for
functions~\cite{gccdoc:attributes}, which allows the programmer to manually mark
them as being pure, so the compiler can take advantage of the information and
reason about the code accordingly. In~\cite{haskell-useless}, Simon Peyton
Jones, a major contributor to the design of the Haskell programming
language~\cite{Simeone}, describes the mutual influence between programming
languages of different paradigms.

\printbibliography

\end{document}

