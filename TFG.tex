\RequirePackage[l2tabu]{nag}
\documentclass[a4paper]{article}
\usepackage[T1]{fontenc}
\usepackage[utf8]{inputenc}
\usepackage{amsmath}
\usepackage{amssymb}
\usepackage{amsthm}
\usepackage{lmodern}
\usepackage{microtype}
\usepackage{tikz}
\usetikzlibrary{cd}
\usepackage{subfiles}
\usepackage{biblatex}
\bibliography{refs.bib}
\usepackage{minted}
\setminted{xleftmargin=1cm}
\usepackage[hidelinks, pdfencoding=auto]{hyperref}

\theoremstyle{plain}
\newtheorem{theorem}{Theorem}[section]
\newtheorem{proposition}[theorem]{Proposition}
\newtheorem{corollary}[theorem]{Corollary}

\theoremstyle{definition}
\newtheorem{definition}[theorem]{Definition}
\newtheorem{notation}[theorem]{Notation}
\newtheorem{example}[theorem]{Example}

\DeclareMathOperator{\Obj}{Obj}
\DeclareMathOperator{\Hom}{Hom}

\newcommand{\id}{\mathrm{id}}
\newcommand{\Set}{\mathrm{Set}}
\newcommand{\Hask}{\ensuremath{\mathrm{Hask}}}
\newcommand{\ev}{\mathrm{ev}}

\newcommand{\cat}[1]{\mathcal{#1}}

\newcommand{\ie}{{i}.{e}., }
\newcommand{\eg}{{e}.{g}. }

\title{Monads in Haskell}
\author{Claudi Lleyda Moltó}
\date{}

\begin{document}
\maketitle
\begin{abstract}
    Upon observing that applying a certain restriction to functions in
    programming, we obtain some substantial benefits but also some important
    apparent limitations, we introduce an abstraction through category theory
    that seems to solve all of the deficits, leaving us in a ``all of the good,
    none of the bad'' situation. We also see how this abstraction can be
    successfully implemented in a real-life programming language, namely
    Haskell.
\end{abstract}
\tableofcontents
\clearpage

\subfile{parts/intro.tex}
\subfile{parts/category-theory.tex}
\subfile{parts/kleisli-triples.tex}
\subfile{parts/adjunctions.tex}
\subfile{parts/haskell.tex}
\subfile{parts/influence.tex}

\printbibliography

\end{document}

