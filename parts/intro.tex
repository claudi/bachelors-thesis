\documentclass[../TGF.tex]{subfiles}

\begin{document}
\section{Motivation}
\label{sec:motivation}
We begin with an observation: functions found in computer programming may not
behave like functions in mathematics.

Unfortunately, it is common to call computer functions and mathematical
functions both functions. In this report we will do so, as it is usually clear
by the context, and only specifying when there might be confusion. In these
situations, we may refer to a computer function as a~\emph{procedure}.

An example of this is the \mintinline{C}{malloc()} function found in
the~\mintinline{C}{stdlib.h} library in the~\mintinline{C}{C} programming
language, which takes a size in number of bits as input and outputs an unique
pointer to an address to a block of memory allocated to fit the requested size,
which is marked as reserved. This function

\begin{enumerate}
    \item generally outputs a different pointer on calls with the same input,
    \item modifies global state by marking allocated memory as reserved.
\end{enumerate}

These properties are incompatible with the definition of a mathematical
function. The first one means it is not well defined, and the second one means
it has side effects, which mathematical functions do not have. This motivates
the following definitions:

\begin{definition}[Pure function]
    We say that a procedure that
    \begin{enumerate}
        \item outputs the same result for the same inputs (\ie is well defined),
        \item does not modify global state.
    \end{enumerate}
    is~\emph{pure}.
    Conversely, we say a procedure that is not pure is~\emph{impure}.
\end{definition}

So, the~\mintinline{C}{malloc()} function is impure, and a function that outputs
the result of squaring an integer is pure.

In theory, a compiler could use its knowledge of certain functions being pure to
better reason about code, which can be used to obtain benefits such as

\begin{enumerate}
    \item If a pure function is called more than once with the
        same input, the result calculated on the previous run can be reused
        without having to call the function again~\cite{frostMemorization}. This
        is called~\emph{memorization}

    \item Since pure functions have no side-effects, it is usually safe to call
        them in parallel \cite{SussPureFunctionParallelization}, leading to
        great optimization.
\end{enumerate}

While pure functions present some significant benefits, they also come with some
apparent drawbacks. For example, it is not obvious how pure functions could
react to user input or output anything to the screen, such as debugging
information or logs. They also seem to be limited to not using global variables.

Most languages circumvent these problems by providing an interface to define
both pure and impure functions. In this way, some of the benefits can be
applied, and the drawbacks can be handled by impure functions.

As of this point, the problems we mentioned are only apparent, and while it is
not at all obvious how they could be avoided, there is also no definitive proof
they cannot. We promise to provide an abstraction that solves these limitations,
and then we will see how it can be successfully implemented into a useful
programming language.
\end{document}
