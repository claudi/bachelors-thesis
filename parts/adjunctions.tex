\documentclass[../TFG.tex]{subfiles}

\begin{document}
\subsection{Adjoint functors}
In category theory we can use functors to compare categories that are not
isomorphic or equivalent. For this we sometimes use functors in both directions.

Adjunctions

This is one of the most important concepts in category theory, and
the definition can be found early on in any book on category theory,
such as [Awodey] or [Leinster]


\begin{definition}[Adjunction]
A pair of functors~\(F:\cat{C}\leftrightarrows\cat{D}:G\) is
an~\emph{adjunction} when for every object~\(C\) in~\(\cat{C}\) and every
object~\(D\) in~\(\cat{D}\) we have a bijection
\[
    \Hom_{\cat{D}}(FD, C) \cong \Hom_{\cat{C}}(D, GC)
\]
This must be natural both in $C$ and in $D$.
\end{definition}

This does not imply that they are inverses to each other, but that there are non
invertible comparison natural transformations
\[
    %\eta: \id_{\cat{C}} \Rightarrow \epsilon \circ \eta
    \eta:1\longrightarrow GF
\]
and
\[
    %\epsilon : \eta \circ\epsilon \Rightarrow \id_{\cat{D}}
    \epsilon:FG\longrightarrow 1
\]
where~\(\eta\) is called the unit for the adjunction and~\(\epsilon\) is called
the counit.

These satisfy the equations
\[
    \begin{tikzcd}
        F \arrow[r, "F\eta", rightarrow] \arrow[dr, equal] & FGF \arrow[d, "\epsilon F", rightarrow] \\
                                                           & F
    \end{tikzcd}
    \qquad
    \begin{tikzcd}
        G \arrow[r, "\eta G", rightarrow] \arrow[dr, equal] & GFG \arrow[d, "G\epsilon", rightarrow] \\
                                                            & G
    \end{tikzcd}
\]
which are called the~\emph{triangle equalities} of adjunctions.

Adjunctions are a common occurrence throughout mathematics. In algebra, the
standard example is the adjunction between vector spaces and sets, where we have
a functor generating a vector space spanned on a set, and another extracting the
underlying set of a vector space. In our context the notion of adjunctions is
important because it generates monads.

\begin{proposition}
    Every adjunction induces a monad.
\end{proposition}
\begin{proof}
    Given an adjunction~\(F:\cat{C}\leftrightarrows\cat{D}:G\) we can define an
    endofunctor by composing~\(F\) and~\(G\) to get~\(T = G \circ
    F:\cat{C}\longrightarrow\cat{C}\). It gets monad structure by
    setting~\(\mu_{A} = G \epsilon_{FA}\) for any object~\(A\) in~\(\cat{C}\)
    and using the unit of the adjunction as unit for the monad.

    It follows from the two equations for unit and counit that $\mu$ and $\eta$
    satisfy the monad laws.
\end{proof}
This is the most important example of monad. One can ask if every monad arises
like this, and the answer is yes.

There are two main constructions of an adjunction to produce a given monad: the
Kleisli construction and the Eilenberg--Moore construction (see [MacLane]).
The Eilenberg--Moore construction seems to be more important in mathematics,
since it is about algebraic structures, but the Kleisli construction has turned
out to be central to theoretical computer science, and it is a main topic in
this work.

%(For example, the Kleisli category construction is not in the basic category
%theory books such as Awodey or Leinster, but it is in the more comprehensive
%text book of Mac Lane.)

\subsection{Kleisli category and adjunctions}
\begin{theorem}
    Given a category~\(\cat{C}\) and a monad~\((T,\eta,\mu)\) on~\(\cat{C}\)
    there exists an adjunction~\(F:\cat{C}\leftrightarrows\cat{C}_{T}:G\)
    such that it induces the monad~\((T,\eta,\mu)\).
\end{theorem}
\begin{proof}
    From the Kleisli category~\(\cat{C}_{T}\), there is a natural
    functor~\(G:\cat{C}_{T}\longrightarrow\cat{C}\) which sends an
    object~\(A\in\cat{C}_{T}\) to~\(TA\in\cat{C}\) and sends a
    morphism~\(f:A\longrightarrow TB\) to the composite
    \[\begin{tikzcd}
        TA \ar[r, "Tf"] & TTB \ar[r, "\mu_{B}"] & TB
    \end{tikzcd}\]
    Indeed, given two morphisms~\(f:A\longrightarrow TB\)
    and~\(g:B\longrightarrow TC\) in~\(\cat{C}_{T}\) we have
    \begin{align*}
        Gg \circ Gf
            &= \mu_{C} \circ Tg \circ \mu_{B} \circ Tf \\
            &= \mu_{C} \circ \mu_{TC} \circ TTg \circ Tf \\
            &= \mu_{C} \circ Tg^{\ast} \circ Tf \\
            &= \mu_{C} \circ Tg \circ_{T} Tf \\
            &= \mu_{C} \circ T(g \circ_{T} f) \\
            &= G(g \circ_{T} f)
    \end{align*}
    and for any object~\(A\) in~\(\cat{C}\) we have
    \[
        G\eta_{A} = \mu_{A} \circ T\eta_{A} = \id_{TA}.
    \]

    Next we define the functor~\(F:\cat{C}\longrightarrow\cat{C}_{T}\)
    sending~\(A\) in~\(\cat{C}\) to~\(A\) in~\(\cat{C}_{T}\), and for sending a
    morphism~\(f:A\longrightarrow B\) in~\(\cat{C}\) to the composite
    in~\(\cat{C}_{T}\) given by
    \[\begin{tikzcd}
        A \ar[r, "f"] & B \ar[r, "\eta_{B}"] & TB
    \end{tikzcd}\]
    We can check that~\(F\) is a functor. If we take~\(f:A\longrightarrow B\)
    and~\(g:B\longrightarrow C\) in~\(\cat{C}\) then
    \begin{align*}
        Fg \circ_{T} Ff
            &= {Fg}^{\ast} \circ Ff \\
            &= (\eta_{C} \circ g)^{\ast} \circ \eta_{B} \circ f \\
            &= \eta_{C} \circ g \circ f \\
            &= F(g \circ f)
    \end{align*}
    and for any object~\(A\) in~\(\cat{C}\) we have
    \[
        F\id_{A} = \eta_{A} \circ_{T} \id_{A} = {\eta_{A}}^{\ast} = \id_{TA}.
    \]

    Let's check that~\(F\) and~\(G\) are adjoint. We first observe
    that~\(GF=T\) by construction. Now we need to provide the unit and
    counit.

    If we define
    \[
        \eta_{A} : A \longrightarrow GF A
    \]
    to be the monad unit and
    \[
        \epsilon_{B} = {\id_{TB}}^{\ast} : FG B \longrightarrow B
    \]
    we get that

%Now we have provided the components of the unit and counit,
%and now we should check that they are natural.

%Finally, now that we know that $\eta$ and $\epsilon$ are natural,
%we should check that they satisfy the uni-counit equations.

%[Here we use the monad laws]

%\bigskip

%Finally we check that this new adjunction we have defined indeed
%induces the original monad $T$. It is clear that the underlying
%endofunctor is $T$, as already observed.

%It is also clear that the unit of the new monad is the unit of
%the adjunction, which in turn is the unit of the original monad $T$.
%So that is fine too.

%Finally we need to check that the multiplication of the new monad
%agrees with the old. The new multiplication is defined in terms of the
%counit for the adjunction, as already explained, and

%in the end we see that the new multiplication coincides with the old.
%So the Kleisli adjunction generates the original monad.
\end{proof}

%All this is explained on the wiki page for Kleisli category
\end{document}
