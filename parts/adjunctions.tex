\documentclass[../TFG.tex]{subfiles}

\begin{document}
\section{Adjoint functors}
As with all strucures in mathematics, categories come along with their own
structure preserving morphism, namely functors. Even though there exists a
notion of an isomorphism of categories (functors with strict inverses), the
natural notion of equality of categories turns out to be a weaker notion, namely
that of equivalence of categories. A yet weaker notion, examples of which abide
in mathematics, is given by adjunctions. They form one of the most imortant
concepts in category theory, and their definition can be found early on in any
book on category, such as~\cite{Awodey} or~\cite{Leinster}.

\subsection{Adjunctions}
\begin{definition}[Adjunction]
A pair of functors~\(F:\cat{C}\leftrightarrows\cat{D}:G\) is
an~\emph{adjunction} when for every object~\(C\) in~\(\cat{C}\) and every
object~\(D\) in~\(\cat{D}\) we have a bijection
\[
    \Hom_{\cat{D}}(FD, C) \cong \Hom_{\cat{C}}(D, GC)
\]
This must be natural both in~\(C\) and in~\(D\).
\end{definition}

This does not imply that they are inverses to each other, but that there are non
invertible comparison natural transformations
\[
    \eta:1\longrightarrow GF
    \qquad
    \text{and}
    \qquad
    \epsilon:FG\longrightarrow 1
\]
where~\(\eta\) is called the~\emph{unit} and~\(\epsilon\) is
called the~\emph{counit} for the adjunction.

These satisfy the equations
\[
    \begin{tikzcd}
        F \arrow[r, "F\eta", rightarrow] \arrow[dr, equal] & FGF \arrow[d, "\epsilon_{F}", rightarrow] \\
                                                           & F
    \end{tikzcd}
    \qquad
    \begin{tikzcd}
        G \arrow[r, "\eta_{G}", rightarrow] \arrow[dr, equal] & GFG \arrow[d, "G\epsilon", rightarrow] \\
                                                              & G
    \end{tikzcd}
\]
which are called the~\emph{triangle equalities} of adjunctions.

Adjunctions are a common occurrence throughout mathematics. A standard example
is the adjunction between vector spaces and sets. Given a set, the first functor
maps this to the vector space spanned by the set. In the other direction, given
a vector space, the second functor forgets the vectors space structure and
returns the underlying set. In our context the notion of adjunctions is
important because its close relation to the notion of monad.

\begin{proposition}
    \label{prop:adjunction-induces-monad}
    Every adjunction induces a monad.
\end{proposition}
\begin{proof}
    Given an adjunction~\(F:\cat{C}\leftrightarrows\cat{D}:G\) we want to show
    that~\((T,\mu,\eta)\) is a monad,
    where~\(T=GF:\cat{C}\longrightarrow\cat{C}\),~\(\mu_{A}=G\epsilon_{FA}\)
    and~\(\eta\) is the unit of the adjunction.

    It follows from the triangle equalities for unit and counit that~\(\mu\)
    and~\(\eta\) satisfy the monad laws.
\end{proof}
This is the most important example of monad. One can ask if every monad arises
like this, and the answer is yes.

Given a monad there are two main constructions of an adjunction to induce a
given monad: the Kleisli construction and the Eilenberg--Moore construction
(see~\cite{maclane}). The Eilenberg--Moore construction seems to be more
important in mathematics, since it is about algebraic structures, but the
Kleisli construction has turned out to be central to theoretical computer
science, and it is a main topic in this work.

\subsection{Kleisli category and adjunctions}
\begin{theorem}
    Given a category~\(\cat{C}\) and a monad~\((T,\eta,\mu)\) on~\(\cat{C}\)
    there exists an adjunction~\(F:\cat{C}\leftrightarrows\cat{C}_{T}:G\)
    such that it induces the monad~\((T,\eta,\mu)\).
\end{theorem}
\begin{proof}
    Throughout this proof we will use~\ref{prop:adjunction-induces-monad} by
    exploiting the equivalence between monads and Kleisli triples, and use their
    properties to prove the statement.

    From the Kleisli category~\(\cat{C}_{T}\), there is a natural
    functor~\(G:\cat{C}_{T}\longrightarrow\cat{C}\) which sends an
    object~\(A\in\cat{C}_{T}\) to~\(TA\in\cat{C}\) and sends a
    morphism~\(f:A\longrightarrow TB\) to the composite
    \[\begin{tikzcd}
        TA \ar[r, "Tf"] & TTB \ar[r, "\mu_{B}"] & TB
    \end{tikzcd}\]
    Indeed, given two morphisms~\(f:A\longrightarrow TB\)
    and~\(g:B\longrightarrow TC\) in~\(\cat{C}_{T}\) we have
    \begin{align*}
        Gg \circ Gf
            &= \mu_{C} \circ Tg \circ \mu_{B} \circ Tf \\
            &= \mu_{C} \circ \mu_{TC} \circ TTg \circ Tf \\
            &= \mu_{C} \circ T\mu_{C} \circ TTg \circ Tf \\
            &= \mu_{C} \circ T(\mu_{C} \circ Tg) \circ Tf \\
            &= \mu_{C} \circ Tg^{\ast} \circ Tf \\
            &= \mu_{C} \circ T(g^{\ast} \circ f) \\
            &= \mu_{C} \circ T(g \circ_{T} f) \\
            &= G(g\circ_{T} f) 
    \end{align*}
    and for any object~\(A\) in~\(\cat{C}\) we have
    \[
        G\eta_{A} = \mu_{A} \circ T\eta_{A} = \id_{TA}
    \]
    by the unit laws of the monad.

    Next we define the functor~\(F:\cat{C}\longrightarrow\cat{C}_{T}\)
    sending~\(A\) in~\(\cat{C}\) to~\(A\) in~\(\cat{C}_{T}\), and for sending a
    morphism~\(f:A\longrightarrow B\) in~\(\cat{C}\) to the composite
    in~\(\cat{C}_{T}\) given by
    \[\begin{tikzcd}
        A \ar[r, "f"] & B \ar[r, "\eta_{B}"] & TB
    \end{tikzcd}\]
    We can check that~\(F\) is a functor. If we take~\(f:A\longrightarrow B\)
    and~\(g:B\longrightarrow C\) in~\(\cat{C}\) then
    \begin{align*}
        Fg \circ_{T} Ff
            &= (Fg)^{\ast} \circ Ff \\
            &= (\eta_{C} \circ g)^{\ast} \circ \eta_{B} \circ f \\
            &= \eta_{C} \circ g \circ f \\
            &= F(g \circ f)
    \end{align*}
    and for any object~\(A\) in~\(\cat{C}\) we have
    \[
        F\id_{A} = \eta_{A} \circ \id_{A} = \eta_{A}
    \]

    Let's check that~\(F\) and~\(G\) are adjoint. We need to provide the unit
    and counit.

    If we define
    \[
        \eta_{A} : A \longrightarrow GF A
    \]
    to be the monad unit and
    \[
        \epsilon_{B} = \id_{TB}
    \]
    we get that
    \[
        f \circ_{T} \epsilon_{A} = \epsilon_{B} \circ_{T} FGf
    \]
    since
    \begin{align*}
        \epsilon_B \circ_T FGf 
            &= \epsilon^{\ast} \circ FGf \\
%(definition of \circ_T)
            &= \id_{TB}^{\ast} \circ (\eta_{TB} \circ \mu_{B} \circ Tf) \\
%(definition of \epsilon, F and G)
            &= (\id_{TB}^{\ast} \circ \eta_{TB}) \circ \mu_{B} \circ Tf \\
%(associativity of \circ)
            &= \id_{TB} \circ \mu_{B} \circ Tf \\
%(Kleisli triple axiom 2)
            &= \mu_{B} \circ Tf \\
            &= f^{\ast} \\
            &= f^{\ast} \circ \id_{TA} \\
%(definition of \epsilon)
            &= f^{\ast} \circ \epsilon_{A} \\
%(definition of \circ_T)
            &= f \circ_{T} \epsilon_{A}
    \end{align*}
    and as such,~\(\eta\) and~\(\epsilon\) are natural.

    Next we must check that they satisfy the triangle equalities of adjunctions.
    Using the Kleisli triple laws we get
    \begin{align*}
        \epsilon_{FA} \circ_{T} F(\eta_{A}) 
            &= \id_{TFA}^{\ast} \circ F(\eta_{A}) \\
            %(definition of \circ_T)
            &= \id_{TA}^{\ast} \circ F(\eta_{A}) \\
            %(since FA = A)
            &= \id_{TA}^{\ast} \circ (\eta_{TA} \circ \eta_{A}) \\
            %(definition of F)
            &= (\id_{TA}^{\ast} \circ \eta_{TA}) \circ \eta_{A} \\
            %(assoc. of \circ)
            &= \id_{TA} \circ \eta_{A} \\
            %(Kleisli triple axiom 2)
            &= \eta_{A} \\
            &= \id_{TA}
    \end{align*}
    and
    \begin{align*}
        G(\epsilon_{A}) \circ \eta_{GA} 
            &= G(\id_{TA}) \circ \eta_{GA} \\
            %(definition of \epsilon)
            &= \mu_{A} \circ T(\id_{TA}) \circ \eta_{GA} \\
            %(definition of G)
            &= \mu_{A} \circ T(\id_{TA}) \circ \eta_{TA} \\
            %(definition of G)
            &= \mu_{A} \circ \id_{TTA} \circ \eta_{TA} \\
            %(T is functor)
            &= \mu_{A} \circ \eta_{TA} \\
            &= \id_{TA}
    \end{align*}

    Finally we check that the monad induced by the
    adjunction~\(F:\cat{C}\leftrightarrows\cat{C}_{T}:G\) recovers the original
    monad. First of on objects~\(GFA = TA\). Similarly one checks that the
    equality holds on morphisms, \ie,~\(GF f = T f\). The induced multiplication
    is given by
    \begin{align*}
        G(\epsilon_{FA}) 
            &= G(\id_{TFA}) \\
            %(definition of \epsilon)
            &= G(\id_{TA}) \\
            %(definition of F)
            &= \mu_{A}
            %(definition of G)
    \end{align*}
    which matches~\ref{prop:adjunction-induces-monad}.
\end{proof}
\end{document}
