\documentclass[../TFG.tex]{subfiles}

\begin{document}
\section{Monads in Category Theory}
\subsection{Introduction to Category Theory}

\begin{definition}[Category]
    \label{def:category}
    A \emph{category}~\(\cat{C}\) is composed of a collection~\(\Obj(\cat{C})\)
    of \emph{objects}, for every two objects~\(A,B\in\Obj(\cat{C})\) a
    set~\(\Hom_{\cat{C}}(A,B)\) of \emph{morphisms} and a \emph{composition
    function}
    \[
        \circ:\Hom_{\cat{C}}(A,B)\times\Hom_{\cat{C}}(B,C) \longrightarrow
        \Hom_{\cat{C}}(A,C)
    \]
    that satisfies
    \begin{enumerate}
        \item \emph{Associativity}: for every three
            morphisms~\(f\in\Hom_{\cat{C}}(A,B)\),~\(g\in\Hom_{\cat{C}}(B,C)\)
            and~\(h\in\Hom_{\cat{C}}(C,D)\)
            we have
            \[
                h \circ (g \circ f) = (h \circ g) \circ f.
            \]
        \item \emph{Left and right units}: for every
            object~\(A\in\Obj(\cat{C})\) there exists a unique
            element~\(\id_{A}\in\Hom_{\cat{C}}(A,A)\)
            such that for every morphism~\(f\in\Hom_{\cat{C}}(A,B)\) we have
            \[
                f \circ \id_{A} = f,
            \]
            and for every morphism~\(g\in\Hom_{\cat{C}}(B,A)\) we have
            \[
                \id_{A} \circ g = g.
            \]
    \end{enumerate}
\end{definition}

\begin{notation}
    We write~\(f:A\longrightarrow B\) or~\(A\overset{f}{\longrightarrow}B\) to
    mean~\(f\in\Hom_{\cat{C}}(A,B)\) and~\(gf\) to mean~\(g\circ f\).
\end{notation}

\begin{example}[Category of sets]
    \label{cat:set}
    There is a category~\(\Set\), where the objects are sets and morphisms are
    maps between sets.
\end{example}

\begin{definition}[Functor]
    \label{def:functor}
    A~\emph{functor}~\(T\) from a category~\(\cat{C}\) to a category~\(\cat{D}\)
    is a map~\(T:\Obj(\cat{C})\longrightarrow\Obj(\cat{D})\), and for every
    pair of objects~\(A,B\in\Obj{C}\) a
    map~\(T:\Hom_{\cat{C}}(A,B)\longrightarrow\Hom_{\cat{D}}(TA,TB)\),
    such that~\(T\) preserves
    \begin{enumerate}
        \item \emph{Composition}: for every two
            morphisms~\(f:A\longrightarrow B\),~\(g:B\longrightarrow C\)
            in~\(\cat{C}\) we have
            \[
                T(g \circ f) = T(g) \circ T(f).
            \]
        \item \emph{Units}: for each object~\(A\in\Obj{C}\) we have
            \[
                T(\id_{A}) = \id_{T(A)}.
            \]
    \end{enumerate}
\end{definition}

\begin{notation}
    We write~\(T:\cat{C}\longrightarrow\cat{D}\) to mean~\(T\) is a functor from
    a category~\(\cat{C}\) to a category~\(\cat{D}\).
    We may also shorten~\(T(A)\) to~\(TA\) or~\(T(f)\) to~\(Tf\), given~\(A\) an
    object in~\(\cat{C}\) and~\(f\) a morphism in~\(\cat{C}\).
\end{notation}

\begin{definition}[Natural transformation]
    \label{def:natural-transformation}
    Let~\(S,T:\cat{C}\longrightarrow\cat{D}\) be functors. A \emph{natural
    transformation}~\(\tau:S\longrightarrow T\) is a
    family~\(\tau=\{\tau_{A}:SA\longrightarrow TA\}_{A\in\Obj(\cat{C})}\) of
    morphisms in~\(\cat{D}\) such that for every
    morphism~\(f:A\longrightarrow B\) in~\(\cat{C}\)
    \[
        Tf \circ \tau_{A} = \tau_{B} \circ Sf,
    \]
    that is, making the following diagram commutative
    \[
        \begin{tikzcd}
            SA \arrow[r, "\tau_{A}"] \arrow[d, "Sf"] & TA \arrow[d, "Tf"] \\
            SB \arrow[r, "\tau_{B}"] & TB
        \end{tikzcd}
    \]
\end{definition}

\subsection{Monads}
\begin{definition}[Monad]
    \label{def:monad}
    A~\emph{monad} over a category~\(\cat{C}\) is a triple~\((T,\eta,\mu)\)
    where~\(T:\cat{C}\longrightarrow\cat{C}\) is a functor
    and~\(\eta:\id\longrightarrow T\) and~\(\mu:TT\longrightarrow T\) are two
    natural transformations such that for every object~\(A\in\Obj(\cat{C})\) the
    diagrams
    \[
        \begin{tikzcd}
            TA \arrow[r, "T\eta_{A}"] & TTA \arrow[d, "\mu_{A}"] & TA \arrow[l,
            swap, "\eta_{TA}"] \\
                                      & TA \arrow[ul, equal] \arrow[ur, equal] &
        \end{tikzcd}
        \qquad\text{and}\qquad
        \begin{tikzcd}
            TTTA \arrow[r, "\mu_{TA}"] \arrow[d, "T\mu_{A}"] & TTA \arrow[d,
            "\mu_{A}"] \\
            TTA \arrow[r, "\mu_{A}"] & TA
        \end{tikzcd}
    \]
    are commutative. This is
    \[
        \mu_{A}\circ T\eta_{A} = \id_{TA} = \mu_{A}\circ\eta_{TA}
        \qquad\text{and}\qquad
        \mu_{A}\circ \mu_{TA}
        = \mu_{A} \circ T\mu_{A}.
    \]
    We call~\(\eta\) and~\(\mu\) the \emph{unit} and the \emph{multiplication}
    of the monad, respectively.

    If~\(\cat{C}\) is the same as~\(\cat{D}\) we say that~\(T\) is
    an~\emph{endofunctor}.
\end{definition}

\begin{example}[Pointed set monad]
    \label{monad:maybe}
    We define the~\emph{pointed set monad}~\((M,\eta,\mu)\) on the category of
    sets as follows.
    \begin{enumerate}
        \item The functor~\(M\) for any object~\(A\) or
            morphism~\(f:A\longrightarrow B\) is defined as
            \begin{gather*}
                MA = A\sqcup\{\bot\}
                \qquad\text{and}\qquad
                \begin{split}
                    Mf:MA&\longrightarrow MB \\
                    \bot&\longmapsto\bot \\
                    A\ni x&\longmapsto f(x)
                \end{split}
            \end{gather*}
        \item The unit~\(\eta\) for any object~\(A\) is defined as the inclusion
            of~\(A\) into~\(MA\).
        \item The multiplication~\(\mu\) for any object~\(A\) is defined as
            \begin{align*}
                \mu_{A}:MMA&\longrightarrow MA \\
                \bot_{1}&\longmapsto \bot \\
                \bot_{2}&\longmapsto \bot \\
                A\ni x&\longmapsto x
            \end{align*}
            where we have denoted~\(MMA=A\sqcup\{\bot_{1}\}\sqcup\{\bot_{2}\}\)
            and~\(MA=A\sqcup\{\bot\}\) for convenience.
    \end{enumerate}
\end{example}

\begin{example}[Power set monad]
    \label{monad:power-set}
    We define the~\emph{power set monad}~\((P,\eta,\mu)\) on the category of
    sets as follows.
    \begin{enumerate}
        \item The functor~\(P\) for any object~\(A\) or
            morphism~\(f:A\longrightarrow B\) is defined as
            \begin{gather*}
                PA = \mathcal{P}(A)
                \qquad\text{and}\qquad
                \begin{split}
                    Pf:PA&\longrightarrow PB \\
                    S&\longmapsto\{f(x)\in B \mid x\in S\}
                \end{split}
            \end{gather*}
        \item The unit~\(\eta\) for any object~\(A\) is defined
            as~\(\eta_{A}(x)=\{x\}\).
        \item The multiplication~\(\mu\) for any object~\(A\) is defined as
            \begin{align*}
                \mu_{A}:PPA&\longrightarrow PA \\
                S&\longmapsto \bigcup_{X\in S}X
            \end{align*}
    \end{enumerate}
\end{example}

\begin{definition}[Cartesian closed category]
    \label{def:product}
    \label{def:cartesian-closed}
    \label{def:exponential object}
    We say that a category~\(\cat{C}\) is~\emph{Cartesian closed} if
    \begin{enumerate}
        \item For any finite, possibly empty, family of
            objects~\(\{A_{i}\}_{i\in I}\) in~\(\cat{C}\) there is an
            object~\(A=\prod_{i\in I}A_{i}\) in~\(\cat{C}\) equipped with
            projection maps.
        \item For any two objects~\(A\) and~\(B\) in~\(\cat{C}\) there is
            object~\(B^{A}\) together with a
            morphism~\(\ev:B^{A}\times A\longrightarrow B\) such that for every
            other object~\(C\) and morphism~\(g:C\times A\longrightarrow B\)
            in~\(\cat{C}\) there is a unique
            morphism~\(\lambda g:C\longrightarrow B^{A}\) such that the diagram
            \[\begin{tikzcd}
                C\times A \arrow[dr, "g"] \arrow[d, "\lambda g\times\id_{A}", swap] & \\
                B^{A}\times A \arrow[r, "\ev"] & B
            \end{tikzcd}\]
            commutes.
    \end{enumerate}
    We say that~\(B^{A}\) is the~\emph{exponential object}, and~\(\times\) is
    the~\emph{product}.
\end{definition}

\begin{example}
    The category of sets is Cartesian closed.
\end{example}
\end{document}
