\documentclass[../TFG.tex]{subfiles}

\begin{document}
\section{The influence of Haskell}
The Haskell language can be regarded as a successful proof of concept on the
usefulness and viability of pure functions, and this has led most traditional
programming languages to adopt some of the features that research on functional
programming has brought to light. Modern programming languages are designed with
support for functional programming or pure functions~\cite{enwiki:1023837642},
and plenty of old programming languages now support the functional programming
paradigm.

For example, the~\mintinline{C}{C} programming language, a famously imperative
and state-full language supports a~\mintinline{C}{pure} attribute for
functions~\cite{gccdoc:attributes}, which allows the programmer to manually mark
them as being pure, so the compiler can take advantage of the information and
reason about the code accordingly. In~\cite{haskell-useless}, Simon Peyton
Jones, a major contributor to the design of the Haskell programming
language~\cite{Simeone}, describes the mutual influence between programming
languages of different paradigms.
\end{document}
