\documentclass[../TFG.tex]{subfiles}

\begin{document}
\section{The Kleisli Triples in Computer Science}
\label{sec:kleisli-triple}
In section~\ref{sec:motivation} we established that pure functions present some
deficits. Let's put category theory aside for a moment and see how we could work
around some of these, which we will then generalize through categorical notions.
\begin{example}[Non-determinism]
    \label{ex:kleisli-non-determinism}
    Suppose we have a procedure~\(A\longrightarrow B\) that has no side-effects
    but is non-deterministic.

    To model its behaviour through a mathematical function we could consider the
    function
    \[
        f:A\longrightarrow\mathcal{P}(B).
    \]
    that outputs all of the possible outputs of said procedure for a single
    output. Since the original procedure had no side-effects, and the new
    function is clearly deterministic, we can assure that~\(f\) is pure, and we
    can reason about it as a mathematical function.

    Given two such functions~\(f:A\longrightarrow\mathcal{P}(B)\)
    and~\(g:B\longrightarrow\mathcal{P}(C)\), we could compose them by
    extending~\(g\) to
    \begin{align*}
        g^{\ast}:\mathcal{P}(B)&\longrightarrow\mathcal{P}(C) \\
        S&\longmapsto\{g(x)\in C \mid x\in S\}
    \end{align*}
    and defining the composition of~\(f\) and~\(g\) to be~\(g^{\ast} \circ f\).
    With the same functions, notice that the composition and~\((-)^{\ast}\)
    satisfy
    \[
        g^{\ast}\circ f^{\ast}
        = (g^{\ast} \circ f)^{\ast}.
    \]

    As a last observation, we see how if we consider the inclusion
    \begin{align*}
        \eta_{A}:A&\longrightarrow\mathcal{P}(A) \\
        a&\longmapsto\{a\}
    \end{align*}
    we get that~\(\eta_{A}^{\ast}=\id_{\mathcal{P}(A)}\) and, with our new
    composition, we get~\(f^{\ast}\circ\eta_{A}=f\) for every
    function~\(f:A\longrightarrow\mathcal{P}(B)\).
\end{example}

\begin{example}[Side-effects]
    \label{ex:kleisli-side-effects}
    Suppose we have a procedure~\(A\longrightarrow B\) that is deterministic but
    presents side-effects.

    Modelling its behaviour through a mathematical function requires a bit more
    effort. A function that has side-effects is one that modifies global state
    beyond its scope. To make it pure it would be enough to also track the
    changes made to global state during its execution. We could achieve this by
    considering the function
    \[
        f':A\times S\longrightarrow B\times S
    \]
    where~\(S\) contains all possible global states, and~\(f'\) now outputs on
    the first component the result of the original procedure, while on the
    second it outputs the effected state.

    Notice how this also models the behaviour of a function that modifies its
    output depending on global state, even if it does not modify it, as the
    global state is part of its input.

    We can also consider the induced pure function
    \begin{align*}
        f:A&\longrightarrow(B\times S)^{S} \\
        a&\longmapsto\bigr(s\mapsto f'(a,s)\bigl)
    \end{align*}
    where~\(B^{A}\) is the exponential object, of which we are implicitly
    assuming existence.

    For convenience we define
    \[
        TA = (A\times S)^{S}.
    \]

    Now, given two functions~\(f:A\longrightarrow TB\)
    and~\(g:B\longrightarrow TC\), we could compose them by, again,
    extending~\(g\) to
    \begin{align*}
        g^{\ast}:TA&\longrightarrow TB \\
        s\mapsto\bigl(g_{1}(s),g_{2}(s)\bigr)&\longmapsto
        s\mapsto\bigl(g\circ g_{1}(s),g_{2}(s)\bigr)
    \end{align*}
    and defining their composition to be~\(g^{\ast} \circ f\).

    We can also define the inclusion
    \begin{align*}
        \eta_{A}:A&\longrightarrow TA \\
        a&\longmapsto\bigl(s\mapsto(a,s)\bigr)
    \end{align*}
    which satisfies~\(\eta_{A}^{\ast}=\id_{TA}\) and, for every
    function~\(f:A\longrightarrow TB\), we see it also
    satisfies~\(f^{\ast}\circ\eta_{A}=f\).
\end{example}

While not being problems inherent to pure functions, there are other situations
commonly found in computer programming that are also foreign to mathematical
functions, which pure functions behave like.

For example, it is difficult to specify the domain and codomain of a procedure,
which often leads to functions that are not defined on all of their input. An
instance of this is the factorial function, which may be written as
\begin{minted}{python}
def factorial(n: int):
    if n == 0:
        return 1
    elif n >= 1:
        return n * factorial(n - 1)
\end{minted}
This implementation of the factorial function accepts an integer as its input,
but it is not defined for negative values. We call this~\emph{partiality}.

\begin{example}[Partiality]
    \label{ex:kleisli-partiality}
    Suppose we have a procedure~\(A\longrightarrow B\) that has no side-effects
    and is deterministic, but instead is partial.

    To model its behaviour we can approach the problem by adding an ``special
    element'', denoted with~\(\bot\) to output in case we hit an input that does
    not have a defined output. This induces a function
    \[
        f:A\longrightarrow B\sqcup\{\bot\}
    \]
    Now, given two such functions~\(f:A\longrightarrow B\sqcup\{\bot\}\)
    and~\(g:B\longrightarrow C\sqcup\{\bot\}\) we can compose them by
    extending~\(g\) to~\(g^{\ast}:B\sqcup\{\bot\}\longrightarrow C\sqcup\{\bot\}\) as
    \begin{align*}
        g^{\ast}:B\sqcup\{\bot\}&\longrightarrow C\sqcup\{\bot\} \\
        b\in B&\longmapsto g(b) \\
        \bot&\longmapsto\bot
    \end{align*}
    and then defining their composition as~\(g^{\ast}\circ f\). With the same
    functions, notice that the composition and~\((-)^{\ast}\) satisfy
    \[
        g^{\ast}\circ f^{\ast} = (g^{\ast}\circ f)^{\ast}.
    \]

    We can also consider the inclusion
    \begin{align*}
        \eta_{A}:A&\longrightarrow A\sqcup\{\bot\} \\
        b&\longmapsto b
    \end{align*}
    and we get that~\(\eta_{A}^{\ast}=\id_{A\sqcup\{\bot\}}\) and, with our previously
    defined composition, we get~\(f^{\ast}\circ\eta_{A}=f\) for every
    function~\(f:A\longrightarrow B\sqcup\{\bot\}\).
\end{example}

Another example are~\emph{exceptions}. Procedures often encounter errors during
their execution, which mathematical functions do not, and must signal them to
other functions that have to deal with them, either by passing the exception up
the call chain (eventually aborting the program) or by directly dealing with it
during their execution.

\begin{example}[Exceptions]
    Suppose we have a procedure~\(A\longrightarrow B\) that has no side-effects
    and is deterministic, but might encounter an error and not be able to
    produce a meaningful result.

    In these situations it is common for functions to return a special value,
    which is not ideal if any output might be valid. More advanced programming
    languages have a system that can manage these errors by separating them from
    normal output. To model the behaviour of these functions, we could consider
    a function
    \[
        f:A\longrightarrow B\sqcup E
    \]
    which, for any execution of the procedure, if the computation is successful
    outputs its result in~\(B\), and if the procedure encounters an error, it
    outputs the error, which would fall in~\(E\).

    The set~\(E\) would be the set error identifiers, such as error codes, and
    we can interpret this induced function as adding information about the
    possible errors encountered to the output of the procedure. The considered
    function~\(f\) is now pure, and we can reason about it mathematically.

    We are once again interested in composing two such
    functions~\(f:A\longrightarrow B\sqcup E\)
    and~\(g:B\longrightarrow C\sqcup E\). For this we could extend~\(g\)
    to~\(g^{\ast}:B\sqcup E\longrightarrow C\sqcup E\) as
    \begin{align*}
        g^{\ast}:B\sqcup E&\longrightarrow C\sqcup E \\
        b\in B&\longmapsto g(b) \\
        e\in E&\longmapsto e
    \end{align*}
    and then defining their composition as~\(g^{\ast}\circ f\). Notice how, with
    the same functions, we can see that
    \[
        g^{\ast} \circ f^{\ast} = (f^{\ast} \circ g)^{\ast}.
    \]

    We can also consider the inclusion
    \begin{align*}
        \eta_{A}:A&\longrightarrow A\sqcup E \\
        a&\longmapsto a
    \end{align*}
    and we get that~\(\eta_{A}^{\ast}=\id_{A\sqcup E}\) and, with our previously
    defined composition, we get~\(f^{\ast}\circ\eta_{A}=f\) for every
    function~\(f:A\longrightarrow B\sqcup E\).
\end{example}
Notice how the partiality example is a particular case of exceptions, \ie
when~\(\#(E) = 1\). We separate them because in the programming context they
have different use cases.

\subsection{Kleisli Triple}
\begin{definition}[Kleisli Triple]
    \label{def:kleisli-triple}
    We define a \emph{Kleisli Triple} over a category~\(\cat{C}\)
    as a triple~\((T, \eta, (-)^{\ast})\) consisting of
    \begin{enumerate}
        \item a class function
            \(T:\Obj(\cat{C})\longrightarrow\Obj(\cat{C})\).
        \item \(\forall A\in\cat{C}\)
            a morphism~\(\eta_{A}:A\longrightarrow TA\)
            in~\(\cat{C}\).
        \item \(\forall f:A\longrightarrow TB\)
            a morphism~\(f^{\ast}:TA\longrightarrow TB\).
    \end{enumerate}
    such that
    \begin{enumerate}
        \item For all~\(A\in\cat{C}\) we have
            \({\eta_{A}}^{\ast} = \id_{TA}:TA\longrightarrow TA\).
        \item For all~\(A,B\in\cat{C}\)
            and \(f:A\longrightarrow TB\)
            the diagram
            \[\begin{tikzcd}
                TA \arrow[r, "f^{\ast}"] & TB \\
                A \arrow[u, "\eta_{A}"] \arrow[ur, "f", swap] &
            \end{tikzcd}\]
            commutes. That is~\(f^{\ast}\circ\eta_{A} = f\).
        \item For all~\(f:A\to TB,g:B\to TC\) we
            have~\(g^{\ast}\circ f^{\ast} = (g^{\ast}\circ f)^{\ast}\).
    \end{enumerate}
\end{definition}

\begin{theorem}
    \label{thm:kleisli-triples-and-monads-correspondence}
    Let~\(\cat{C}\) be a category.
    There is a bijective correspondence between Kleisli Triples over~\(\cat{C}\)
    and monads over~\(\cat{C}\).
\end{theorem}
\begin{proof}
    Let~\((T, \eta, (-)^{\ast})\) be a Kleisli triple over~\(\cat{C}\) and set
    \begin{equation}
        \label{eq:monad-unit-in-kleisli-trip}
        \begin{split}
            T:\cat{C} & \longrightarrow\cat{C} \\
            A & \longmapsto TA \\
            f:A\rightarrow B & \longmapsto
            (\eta_{B}\circ f)^{\ast}:TA\rightarrow TB
        \end{split}
    \end{equation}
    and
    \begin{equation}
        \label{eq:monad-prod-in-kleisli-trip}
        \mu_{A} = (\id_{TA})^{\ast}.
    \end{equation}
    We will see that~\((T,\eta,\mu)\) is a monad.

    Let's check that~\(T\) is a functor.
    For every two~\(f:A\longrightarrow B\)
    and~\(g:B\longrightarrow C\)
    morphisms in~\(\cat{C}\)
    we have
    \begin{align*}
        Tg\circ Tf &= (\eta_{C}\circ g)^{\ast}\circ
                      (\eta_{B}\circ f)^{\ast} \\
                   &= \bigl((\eta_{C}\circ g)^{\ast}\circ
                      (\eta_{B}\circ f)\bigr)^{\ast} \\
                   &= \bigl((\eta_{C}\circ g)^{\ast}\circ
                      \eta_{B}\circ f\bigr)^{\ast} \\
                   &= (\eta_{C}\circ g\circ f)^{\ast} \\
                   &= T(g\circ f)
    \end{align*}
    and for all~\(A\in\cat{C}\),
    \[
        T\id_{A} = (\eta_{A}\circ \id_{A})^{\ast} = \eta_{A}^{\ast} = \id_{TA},
    \]
    which proves that~\(T\) is a functor.

    Next we want to see that~\(\mu=\{\mu_{A}\}_{A\in\Obj(\cat{C})}\) is a
    natural transformation.
    Given morphism~\(f:A\longrightarrow B\) in~\(\cat{C}\) we want to prove that
    \[
        Tf \circ \mu_{A} = \mu_{B} \circ TTf.
    \]
    Using~\eqref{eq:monad-prod-in-kleisli-trip} we get
    \begin{gather*}
        \begin{split}
            Tf \circ \mu_{A} &= Tf \circ (\id_{TA})^{\ast} \\
                &= (\eta_{B} \circ f)^{\ast} \circ (\id_{TA})^{\ast} \\
                &= \bigl((\eta_{B} \circ f)^{\ast} \circ \id_{TA}\bigr)^{\ast} \\
                &= \bigl((\eta_{B} \circ f)^{\ast}\bigr)^{\ast} \\
                &= (Tf)^{\ast}
        \end{split}
        \qquad\text{and}\qquad
        \begin{split}
            \mu_{B} \circ TTf &= (\id_{TB})^{\ast} \circ TTf \\
                &= (\id_{TB})^{\ast} \circ (\eta_{TB} \circ Tf)^{\ast} \\
                &= \bigl((\id_{TB})^{\ast} \circ \eta_{TB} \circ Tf\bigr)^{\ast} \\
                &= \bigl(\id_{TB} \circ Tf\bigr)^{\ast} \\
                %&= \bigl(T\id_{B} \circ Tf\bigr)^{\ast} \\
                &= (Tf)^{\ast}
        \end{split}
    \end{gather*}
    and we have proved that~\(\mu\) is a natural transformation.

    We can check that~\(\eta=\{\eta_{A}\}_{A\in\Obj(\cat{C})}\) is also a
    natural transformation.
    By using~\eqref{eq:monad-unit-in-kleisli-trip} we can see that, for any
    morphism~\(f:A\longrightarrow B\) in~\(\cat{C}\), we have
    \[
        Tf \circ \eta_{A}
        = (\eta_{B}\circ f)^{\ast} \circ \eta_{A}
        = \eta_{B} \circ f,
    \]
    which shows that~\(\mu\) is a natural transformation.

    Finally, we can check that~\(\mu\) and~\(\eta\) satisfy the monad laws
    \[
        \mu_{A}\circ T\eta_{A} = \id_{TA} = \mu_{A}\circ\eta_{TA}
        \qquad\text{and}\qquad
        \mu_{A}\circ \mu_{TA}
        = \mu_{A} \circ T\mu_{A}.
    \]
    Using the Kleisli triple laws calculate
    \begin{align*}
        \mu_{A} \circ T\eta_{A}
            &= \mu_{A} \circ (\eta_{TA} \circ \eta_{A})^{\ast} \\
            &= {\id_{TA}}^{\ast} \circ (\eta_{TA} \circ \eta_{A})^{\ast} \\
            &= ({\id_{TA}}^{\ast} \circ \eta_{TA} \circ \eta_{A})^{\ast} \\
            &= \bigl(({T\id_{A}}^{\ast} \circ \eta_{TA}) \circ \eta_{A}\bigr)^{\ast} \\
            &= (T\id_{A} \circ \eta_{A})^{\ast} \\
            &= {\eta_{A}}^{\ast} \\
            &= \id_{TA} \\
            &= {\id_{TA}}^{\ast} \circ \eta_{TA} \\
            &= \mu_{A} \circ \eta_{TA}
    \end{align*}
    and
    \begin{align*}
        \mu_{A} \circ \mu_{TA}
            &= {\id_{TA}}^{\ast} \circ {\id_{TTA}}^{\ast} \\
            &= ({\id_{TA}}^{\ast} \circ \id_{TTA})^{\ast} \\
            &= {\id_{TA}}^{\ast\ast} \\
            &= \bigl(\id_{TA} \circ (\id_{TA})^{\ast}\bigr)^{\ast} \\
            &= \bigl({\id_{TA}}^{\ast} \circ \eta_{TA}\circ(\id_{TA})^{\ast}\bigr)^{\ast} \\
            &= {\id_{TA}}^{\ast} \circ \bigl(\eta_{TA}\circ(\id_{TA})^{\ast}\bigr)^{\ast} \\
            &= {\id_{TA}}^{\ast} \circ T(\id_{TA})^{\ast} \\
            &= \mu_{A} \circ T\mu_{A}
    \end{align*}
    and this proves that a Kleisli Triple induces a monad.

    Let~\((T,\eta,\mu)\) be a monad over~\(\cat{C}\) and for any
    morphism~\(f:A\longrightarrow TB\) in~\(\cat{C}\) set
    \begin{equation}
        \label{eq:kleisli-ast-in-monad}
        f^{\ast} = \mu_{B} \circ Tf.
    \end{equation}
    We want to check that~\((T,\eta,(-)^{\ast})\) is a Kleisli triple.

    By the monad unit law we have
    \[
        {\eta_{TA}}^{\ast} = \mu_{TA} \circ T\eta_{TA} = \id_{TA},
    \]
    and for any two morphisms~\(f:A\longrightarrow TB\)
    and~\(g:B\longrightarrow TC\) in~\(\cat{C}\) we have
    \begin{align*}
        g^{\ast}\circ f^{\ast} &= \mu_{C} \circ Tg \circ \mu_{B} \circ Tf \\
                               &= \mu_{C} \circ \mu_{TC} \circ TTg \circ Tf \\
                               &= \mu_{C} \circ T\mu_{C} \circ TTg \circ Tf \\
                               &= \mu_{C} \circ T(\mu_{C} \circ Tg \circ f) \\
                               &= \mu_{C} \circ T(g^{\ast} \circ f) \\
                               &= (g^{\ast} \circ f)^{\ast}
    \end{align*}

    Then, for any morphism~\(f:A\longrightarrow TB\) in~\(\cat{C}\) we can use
    the unit law to get
    \begin{align*}
        f^{\ast} \circ \eta_{A}
            &= \mu_{B} \circ Tf \circ \eta_{A} \\
            &= \mu_{B} \circ \eta_{TB} \circ f \\
            &= \id_{TB} \circ f \\
            &= f
    \end{align*}
    and this proves that every monad induces a Kleisli Triple.

    It is clear that these two constructions are inverse to each other.
\end{proof}

\subsection{The Kleisli category}
For this definition to be useful we must also show that the defined morphisms
are well behaved in some sense. At the very minimum, we should be able to
compose them and there should be a unit function.

We show that they actually form a category, and give an explicit formula for
their composition and units.

\begin{proposition}[The Kleisli category]
    Given a Kleisli triple~\((T,\eta,(-)^{\ast})\) over a category~\(\cat{C}\),
    the following data forms a category~\(\cat{C}_{T}\), which we call
    the~\emph{Kleisli category}.
    \begin{enumerate}
        \item The objects~\(\Obj(\cat{C}_{T}) = \Obj(\cat{C})\).
        \item For any two objects~\(A,B\) in~\(\cat{C}_{T}\), the morphisms
            \[
                \Hom_{\cat{C}_{T}}(A,B) = \Hom_{C}(A,TB).
            \]
        \item For any two
            morphisms~\(f:A\longrightarrow TB\),~\(g:B\longrightarrow TC\)
            in~\(\cat{C}_{T}\), the composition
            \[
                g \circ_{T} f = g^{\ast} \circ f.
            \]
    \end{enumerate}
\end{proposition}
\begin{proof}
    We start by checking that the composition is associative. For any three
    morphisms~\(f:A\longrightarrow TB\),~\(g:B\longrightarrow TC\)
    and~\(h:C\longrightarrow TD\) in~\(\cat{C}_{T}\) we have
    \begin{align*}
        h \circ_{T} (g \circ_{T} f)
            &= h^{\ast} \circ (g \circ_{T} f) \\
            &= h^{\ast} \circ (g^{\ast} \circ f) \\
            &= (h^{\ast} \circ g^{\ast}) \circ f \\
            &= (h^{\ast} \circ g)^{\ast} \circ f \\
            &= (h^{\ast} \circ g) \circ_{T} f \\
            &= (h \circ_{T} g) \circ_{T} f,
    \end{align*}
    were we have used the third property of Kleisli triples.

    We can also see that for every object~\(A\) in~\(\Obj(\cat{C}_{T})\) there
    is a unit morphism~\(\id_{A}\). If we take
    \[
        \id_{A} = \eta_{A}
    \]
    then for every morphism~\(f:A\longrightarrow TB\) in~\(\cat{C}_{T}\) we have
    \begin{align*}
        f \circ_{T} \id_{A}
        &= f \circ_{T} \eta_{A} \\
        &= f^{\ast} \circ \eta_{A} \\
        &= f
    \end{align*}
    and for every morphism~\(g:B\longrightarrow TA\) in~\(\cat{C}_{T}\) we have
    \begin{align*}
        \id_{A} \circ_{T} g
        &= \eta_{A} \circ_{T} g \\
        &= \eta_{A}^{\ast} \circ g \\
        &= \id_{TA} \circ g \\
        &= g
    \end{align*}
    by using the first and second properties of Kleisli triples.

    This is enough to check that the construction satisfies the definition of a
    category given in~\ref{def:category}.
\end{proof}
\end{document}
