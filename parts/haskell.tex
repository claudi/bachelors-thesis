\documentclass[../TFG.tex]{subfiles}

\begin{document}
\section{Monads in Haskell}
So far we have described functional programming at a theoretical level. We
emphasized its apparent drawbacks and proposed an abstract solution for them
through the use of Kleisli triples in section~\ref{sec:kleisli-triple}, but we
are interested in how this solution can be implemented, since abstractions might
not always be useful or feasible in practice.

For this we use Haskell, a language that makes use of these abstract concepts to
be a programming language where programmers are only allowed to use pure
functions by default. Such languages are called~\emph{purely functional} or said
to belong to the~\emph{purely functional programming
paradigm}~\cite{paradigms-overview}.

\subsection{Haskell}
To achieve this, the Haskell language makes use of notation that is atypical
outside of such languages. In this section we aim to introduce the basics of it
so that we can then use it to explain how the theory we discussed previously is
applied. More thorough introductions can be found in~\cite{haskell-org-docs}.

\subsubsection{Types and functions}
In Haskell, any element belongs to a particular~\emph{type}. Examples of types
are~\mintinline{Haskell}{Int} and~\mintinline{Haskell}{Char}, used for integer
numbers and characters respectively. To denote an element belonging to a certain
type we write
\begin{minted}{Haskell}
x :: Int
\end{minted}
which reads ``\mintinline{Haskell}{x} belongs to the~\mintinline{Haskell}{Int}
type''.
The~\mintinline{Haskell}{::} symbol can be seen as the~\(\in\) symbol in math,
and types may be regarded as sets for the most part.

There are also types of functions. For any two types~\mintinline{Haskell}{a}
and~\mintinline{Haskell}{b} there is a type
\begin{minted}{Haskell}
a -> b
\end{minted}
whose elements are functions from the first type to the second. This means that
functions in Haskell take only one argument, which is written after the function
name, without parenthesis and separated by spaces. For example, if we have
a~\mintinline{Haskell}{factorial :: Int -> Int} function and we want to evaluate
it at~\mintinline{Haskell}{5 :: Int} we write
\begin{minted}{Haskell}
factorial 5
\end{minted}

Concrete types like~\mintinline{Haskell}{Int} and~\mintinline{Haskell}{Char} are
written starting with an uppercase letter, while generic types are written in
lowercase.

Of course, since the function types are types, we can have functions that take
or output other functions. An example would be
\begin{minted}{Haskell}
f :: a -> (b -> c)
g :: (a -> b) -> c
\end{minted}
Here~\mintinline{Haskell}{f} is a function that takes an element of
type~\mintinline{Haskell}{a} and returns an element of
type~\mintinline{Haskell}{b -> c}, \ie a function from~\mintinline{Haskell}{b}
to~\mintinline{Haskell}{c}, while~\mintinline{Haskell}{g} is a function that
takes a function from~\mintinline{Haskell}{a} to~\mintinline{Haskell}{b} as an
argument an returns an element of type~\mintinline{Haskell}{c}. If we
regard~\mintinline{Haskell}{->} as an operator, we say it is right associative,
and we can write
\begin{minted}{Haskell}
f :: a -> b -> c
\end{minted}
instead of
\begin{minted}{Haskell}
f :: a -> (b -> c)
\end{minted}

We can use this simulate functions with multiple arguments.
If we consider a function~\mintinline{Haskell}{sum :: Int -> Int -> Int} that
adds two integers and two elements~\mintinline{Haskell}{41 :: Int}
and~\mintinline{Haskell}{19 :: Int} we would have
\begin{minted}{Haskell}
sum       :: Int -> Int -> Int -- 1
sum 41    :: Int -> Int        -- 2
sum 41 19 :: Int               -- 3
\end{minted}
where
\begin{enumerate}
    \item The~\mintinline{Haskell}{sum :: Int -> Int -> Int} function takes an
        argument of type~\mintinline{Haskell}{Int} and outputs a function of
        type~\mintinline{Haskell}{Int -> Int}.
    \item After evaluating~\mintinline{Haskell}{sum :: Int -> Int -> Int}
        on~\mintinline{Haskell}{41 :: Int} we get a
        function~\mintinline{Haskell}{Int -> Int},
        denoted~\mintinline{Haskell}{sum 41}.
        Intuitively,~\mintinline{Haskell}{sum 41} is a function that takes an
        integer and returns the result of adding~\(41\) to that integer.
    \item We can evaluate the function~\mintinline{Haskell}{sum 41}
        on~\mintinline{Haskell}{19 :: Int} to get a final result of
        type~\mintinline{Haskell}{Int}.
\end{enumerate}

We can also define operations, or infix functions. To do this we surround the
name of the function with parentheses as such
\begin{minted}{Haskell}
(+) :: Int -> Int -> Int
\end{minted}
and we can later write
\begin{minted}{Haskell}
1 + 2 :: Int
\end{minted}

\subsubsection{Function bodies}
To explain how to define the body of a function we start with an example
\begin{minted}{Haskell}
square :: Int -> Int -- 1
square n = n * n     -- 2
\end{minted}
where
\begin{enumerate}
    \item We first declare that we will define a
        function~\mintinline{Haskell}{square} of
        type~\mintinline{Haskell}{Int -> Int}.
    \item Here we define the behaviour of our function given an
        argument~\mintinline{Haskell}{n}. The notation is similar to the one
        we use in math, where we write \(f(n) = n*n\).
\end{enumerate}
Another example
\begin{minted}{Haskell}
multiply :: Int -> Int -> Int
multiply n m = n * m
\end{minted}
although functions in Haskell technically only take one argument, we can define
a function that takes multiple parameters in one line as shown in the example
above.

Definitions like
\begin{minted}{Haskell}
multiplyByTwo :: Int -> Int
multiplyByTwo = multiply 2
\end{minted}
are also valid. Here the function~\mintinline{Haskell}{multiplyByTwo} is defined
in terms of the function~\mintinline{Haskell}{multiply 2}.

\subsubsection{Algebraic data types}
We can also define new types from existing ones. We already know how to do this
with~\mintinline{Haskell}{->}, but we can also use
the~\mintinline{Haskell}{data} keyword
\begin{minted}{Haskell}
data Complex = Cartesian Double Double
\end{minted}
which consists of a space-separated blueprint~\mintinline{Haskell}{Cartesian
Double Double} for the new~\mintinline{Haskell}{Complex} type.

This defines a new called~\mintinline{Haskell}{Complex}. This definition comes
with an associated
map~\mintinline{Haskell}{Cartesian :: Double -> Double -> Complex} which lets us
instantiate elements in the~\mintinline{Haskell}{Cartesian} type, \ie
\begin{minted}{Haskell}
Cartesian 1.0 2.0 :: Complex
\end{minted}

Here we constructed a type~\mintinline{Haskell}{Complex} from
two~\mintinline{Haskell}{Double} types. We can see this as the Cartesian product
of two types, which in math we would
write~\(\mathbb{C}=\mathbb{R}\times\mathbb{R}\).

We can also define new types in a similar fashion to the union of sets with
\begin{minted}{Haskell}
data Shape = Circle Double | Polygon Int Double
\end{minted}
where~\mintinline{Haskell}{|} reads as ``or''. Here, a member of
the~\mintinline{Haskell}{Shape} type could be either
a~\mintinline{Haskell}{Circle}, specified by the length of
type~\mintinline{Haskell}{Double} of its radius or a
(regular)~\mintinline{Haskell}{Polygon}, specified the number of sides of
type~\mintinline{Haskell}{Int} and its apothem length, also of
type~\mintinline{Haskell}{Double}.

So far we are capable of defining new types by addition and multiplication. We
call the first kind of types~\emph{sum types} and the second type~\emph{product
types}. Together it is said that the types form an algebraic structure, and
types made combining these operations are~\emph{algebraic data types}.

The definition of~\mintinline{Haskell}{Shape} induces two functions
\begin{minted}{Haskell}
Circle  :: Double -> Shape
Polygon :: Int -> Double -> Shape
\end{minted}
but these words, in this case~\mintinline{Haskell}{Circle}
and~\mintinline{Haskell}{Polygon}, acquire another meaning when defying a new
type. To implement a function to calculate the area of our shapes we would write
\begin{minted}{Haskell}
area :: Shape -> Double
area (Circle r)    = pi*r*r
area (Polygon n a) = n*a*a*tan(pi/n)
\end{minted}
which introduces the other use of the words~\mintinline{Haskell}{Circle}
and~\mintinline{Haskell}{Polygon} we used to define~\mintinline{Haskell}{Shape}.
They are used for~\emph{pattern matching}. To evaluate
the~\mintinline{Haskell}{area} function on an element we must
first be able to identify what kind of~\mintinline{Haskell}{Shape} it is. This
is done through matching the instance of said element on the constructors, and
the program will make sure to execute the adequate one.

Pattern matching also applies to the previous example:
\begin{minted}{Haskell}
conjugate :: Complex -> Complex
conjugate (Cartesian x y) = Cartesian x (-y)
\end{minted}
notice here the use of~\mintinline{Haskell}{Cartesian} to pattern match and to
construct an element of the~\mintinline{Haskell}{Complex} type.

We do not always have to pattern match. We can avoid it when we can treat the
different options equally and we do not need to access the parameters used to
construct the instance of the type, for example
\begin{minted}{Haskell}
sameArea :: Shape -> Shape -> Bool
sameArea s1 s2 = (area s1) == (area s2)
\end{minted}

\subsubsection{Parametrized data types}
So far we have seen how we can define types from other concrete types. The
Haskell language also allows us to define types parametrized by other types, for
example we can define a binary tree type to hold nodes of any type as
\begin{minted}{Haskell}
data Pair a = Tuple a a
\end{minted}
Here~\mintinline{Haskell}{a} is a generic type, and acts as a parameter to the
type definition. We can then instantiate a pair
of~\mintinline{Haskell}{Int} elements as
\begin{minted}{Haskell}
Tuple 2 3 :: Pair Int
\end{minted}

Examples of generic functions over the type are
\begin{minted}{Haskell}
first :: Pair a -> a
first (Tuple x _) = x

second :: Pair a -> a
second (Tuple _ y) = y

flip :: Pair a -> Pair a
flip (Tuple x y) = Tuple y x

apply :: (a -> a -> b) -> Pair a -> b
apply f (Tuple x y) = f x y
\end{minted}
here we have used~\mintinline{Haskell}{_} for the unused parameters
in~\mintinline{Haskell}{first} and~\mintinline{Haskell}{second}. This is because
they are clearly not used in the function body, and if we assign a variable name
to an unused argument the Haskell compiler will throw an error.

Generic types can be parametrized by more than one type. We can generalize
the~\mintinline{Haskell}{Pair} into
\begin{minted}{Haskell}
data Pair a b = Tuple a b
\end{minted}
and then we would have
\begin{minted}{Haskell}
Tuple 3 'c' :: Pair Int Char

first :: Pair a b -> a
first (Tuple x _) = x

second :: Pair a b -> b
second (Tuple _ y) = y

flip :: Pair a b -> Pair b a
flip (Tuple x y) = Tuple y x

apply :: (a -> b -> c) -> Pair a b -> c
apply f (Tuple x y) = f x y
\end{minted}

\subsection{The \texorpdfstring{\mintinline{Haskell}{Monad}}{Monad} typeclass}
The Haskell language allows users to define~\emph{typeclasses}, which area set
of constraints we give to a certain set of types.
They are similar to interfaces, which appear in other languages.

For our purposes, we can picture them as a definition. A typeclass describes a
behavior we expect.

Let's see an example. The~\mintinline{Haskell}{Functor} typeclass is
\begin{minted}{Haskell}
class Functor f where
    fmap :: (a -> b) -> f a -> f b
\end{minted}
From the usage of~\mintinline{Haskell}{f}, the Haskell compiler can infer
that it must be a parametric type with one parameter, and
with this information we can read the previous statement as
\begin{quote}
    For a parametric datatype~\mintinline{Haskell}{f} with one parameter to be
    called a Functor, it must be equipped with a structure

    \mintinline{Haskell}{fmap :: (a -> b) -> f a -> f b}
\end{quote}

Let's see an example of an implementation. If we define the parametric data
type~\mintinline{Haskell}{Maybe} as follows
\begin{minted}{Haskell}
data Maybe a = Just a | None
\end{minted}
we can make it into a Functor by writing
\begin{minted}{Haskell}
instance Functor Maybe where
    fmap :: (a -> b) -> Maybe a -> Maybe b
    fmap _ Nothing  = Nothing
    fmap f (Just x) = Just (f x)
\end{minted}
where we first declare that we want to show that~\mintinline{Haskell}{Maybe a}
is a~\mintinline{Haskell}{Functor}, which we do by providing a definition
for~\mintinline{Haskell}{fmap :: (a -> b) -> Maybe a -> Maybe b}.

Given a parametric datatype~\mintinline{Haskell}{f a}, the functor typeclass
provides a function that can be understood as having
type~\mintinline{Haskell}{fmap :: (a -> b) -> (f a -> f b)}, which reminds us of
the property of functors to act on functions. Additionally, the parametric
datatype itself acts on types, by transforming an existing type into a new one,
behaviour similar to that of a functor.

This is not enough for a function on objects and morphisms to be a functor. The
functor laws must be satisfied as well. The Haskell language does not provide a
way to express such laws, since for this it would need a way to write and
validate proofs, which would add too much complexity to the language. The
compiler assumes any typeclass implementations it encounters satisfy such
properties, which are only generally found in external documentation. Failing to
satisfy these laws could lead to buggy code and unintended behaviour.

We will not prove typeclass laws for specific instances we provide in this
section, as the proofs are usually straightforward but uncomfortable to write.

The laws for the functor typeclass are
\begin{minted}{Haskell}
fmap id = id
fmap (f . g) = (fmap f . fmap g)
\end{minted}
that are clearly equivalent to the functor laws found in~\ref{def:functor}.
The~\mintinline{Haskell}{f} that appears in these laws is different from
the~\mintinline{Haskell}{f} found in the definition of
the~\mintinline{Haskell}{Functor} typeclass, since it now corresponds to an
arbitrary function. This misleading use of notation happens often in Haskell,
and will be repeated in further typeclass laws in this document.

There is a notion of~\emph{inheritance} of typeclasses. Typeclass inheritance is
when a typeclass has a superclass. This is a way of expressing that a typeclass
requires another typeclass to be available for a given type before you can write
an instance. For example, Haskell provides a class that inherits
from~\mintinline{Haskell}{Functor}
\begin{minted}{Haskell}
class Functor f => Applicative f where
    (<*>) :: f (a -> b) -> f a -> f b
    pure :: a -> f a
\end{minted}
the~\mintinline{Haskell}{=>} notation is unfortunate, as in math it is usually
associated with an implication.

In this situation, to prove that a datatype is
an~\mintinline{Haskell}{Applicative}, we must first prove it is
a~\mintinline{Haskell}{Functor}, and then provide implementations for the two
required functions.
The laws for the~\mintinline{Haskell}{Applicative} typeclass are
\begin{minted}{Haskell}
pure id <*> v = v
pure (.) <*> u <*> v <*> w = u <*> (v <*> w)
pure f <*> pure x = pure (f x)
u <*> pure y = pure ($ y) <*> u
\end{minted}
and we can make~\mintinline{Haskell}{Maybe}, which we know to be
a~\mintinline{Haskell}{Functor}, into an~\mintinline{Haskell}{Applicative}
\begin{minted}{Haskell}
instance Applicative Maybe where
    (<*>) :: Maybe (a -> b) -> Maybe a -> Maybe b
    (<*>) (Just f) m = fmap f m
    (<*>) _ _ = Nothing

    pure :: a -> Maybe a
    pure = Just
\end{minted}
this class coincides with the notion of lax monoidal functor, which is not
related to our topic. They are used to define the~\mintinline{Haskell}{Monad}
typeclass.

\begin{minted}{Haskell}
class Applicative m => Monad m where
    (>>=) :: m a -> (a -> m b) -> m b
    (>>) :: m a -> m b -> m b
    return :: a -> m a
\end{minted}
where~\mintinline{Haskell}{(>>=)} is called~\emph{bind}.
Any~\mintinline{Haskell}{Monad} implementation must satisfy
\begin{minted}{Haskell}
return a >>= k = k a
m >>= return = m
m >>= (\x -> k x >>= h) = (m >>= k) >>= h

k >> f = k >>= \_ -> f

pure = return
m1 <*> m2 = m1 >>= (x1 -> m2 >>= (x2 -> return (x1 x2)))
\end{minted}
finally, we can make~\mintinline{Haskell}{Maybe}, which we know to be
an~\mintinline{Haskell}{Applicative}, into a~\mintinline{Haskell}{Monad}
\begin{minted}{Haskell}
instance Monad Maybe where
    (>>=) :: Maybe a -> (a -> Maybe b) -> Maybe b
    Just x  >>= k = k x
    Nothing >>= _ = Nothing

    (>>) :: Maybe a -> Maybe b -> Maybe b
    Just _  >> m = m
    Nothing >> _ = Nothing

    return :: a -> Maybe a
    return = pure
\end{minted}

\subsection{Haskell monads are monads}
The~\mintinline{Haskell}{Monad} structure defined by Haskell
\begin{minted}{Haskell}
class Applicative m => Monad m where
    (>>=) :: m a -> (a -> m b) -> m b
    (>>) :: m a -> m b -> m b
    return :: a -> m a
\end{minted}
subject to the laws
\begin{minted}{Haskell}
return a >>= k = k a
m >>= return = m
m >>= (\x -> k x >>= h) = (m >>= k) >>= h

k >> f = k >>= \_ -> f

pure = return
m1 <*> m2 = m1 >>= (x1 -> m2 >>= (x2 -> return (x1 x2)))
\end{minted}
does not correspond, at first sight, to the definition of monad given
in~\ref{def:monad}.

We begin observing that the~\mintinline{Haskell}{Monad}
operation~\mintinline{Haskell}{(>>)} can be uniquely determined in terms
of~\mintinline{Haskell}{(>>=)}, the bind operation.
This means that we can omit this operation from our reasoning. In fact, it is
only defined because it is useful in a programming setting, and bears no
particular mathematical relevance to us.

Our next observation lies on the bind operation. If we flip its arguments, we
get an operation with type signature
\begin{minted}{Haskell}
(a -> m b) -> (m a -> m b)
\end{minted}
which coincides with the~\((-)^{\ast}\) operation, while the type signature for
the~\mintinline{Haskell}{return} function coincides with~\(\eta_{A}\)
(where~\(A\) is determined by context in the Haskell program), as defined in the
definition of Kleisli triple in~\ref{def:kleisli-triple}.

With this correspondence we can now prove that the~\mintinline{Haskell}{Monad}
structure defined by Haskell is a monad by first showing that it is a Kleisli
triple, and then using
theorem~\ref{thm:kleisli-triples-and-monads-correspondence}. This can be shown
directly from the first three~\mintinline{Haskell}{Monad} laws.
\end{document}
