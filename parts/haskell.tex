\documentclass[../TFG.tex]{subfiles}

\begin{document}
\section{Monads in Haskell}
So far we have described functional programming at a theoretical level. We
emphasized its apparent drawbacks and proposed an abstract solution for them
through the use of Kleisli triples in section~\ref{sec:kleisli-triple}, but we
are interested in how this solution can be implemented, since abstractions might
not always be useful or feasible in practice.

For this we use Haskell, a language that makes use of these abstract concepts to
be a programming language where programmers are only allowed to use pure
functions by default. Such languages are called~\emph{purely functional} or said
to belong to the~\emph{purely functional programming
paradigm}~\cite{paradigms-overview}.

\subsection{Haskell}
To achieve this, the Haskell language makes use of notation that is atypical
outside of such languages. In this section we aim to introduce the basics of it
so that we can then use it to explain how the theory we discussed previously is
applied. More thorough introductions can be found in~\cite{haskell-org-docs}.

\subsubsection{Types and functions}
In Haskell, any element belongs to a particular~\emph{type}. Examples of types
are~\mintinline{Haskell}{Int} and~\mintinline{Haskell}{Char}, used for integer
numbers and characters respectively. To denote an element belonging to a certain
type we write
\begin{minted}{Haskell}
x :: Int
\end{minted}
which reads ``\mintinline{Haskell}{x} belongs to the~\mintinline{Haskell}{Int}
type''. There are a lot of types, but these are some of the most basic ones
which we will be using throughout the document.
\begin{minted}{Haskell}
1   :: Int
'a' :: Char
1.0 :: Double
\end{minted}
The~\mintinline{Haskell}{::} symbol can be seen as the~\(\in\) symbol in math,
and types may be regarded as sets for the most part.

There are also types of functions. For any two types~\mintinline{Ocaml}{a}
and~\mintinline{Ocaml}{b} there is a type
\begin{minted}{Ocaml}
a -> b
\end{minted}
whose elements are functions from the first type to the second. This means that
functions in Haskell take only one argument, which is written after the function
name, without parenthesis and separated by spaces. For example, if we have
a~\mintinline{Haskell}{factorial :: Int -> Int} function and we want to evaluate
it at~\mintinline{Haskell}{5 :: Int} we write
\begin{minted}{Haskell}
factorial 5
\end{minted}

Concrete types like~\mintinline{Haskell}{Int} and~\mintinline{Haskell}{Char} are
written starting with an uppercase letter, while generic types are written in
lowercase.

Of course, since the function types are types, we can have functions that take
or output other functions. An example would be
\begin{minted}{Haskell}
f :: a -> (b -> c)
g :: (a -> b) -> c
\end{minted}
Here~\mintinline{Haskell}{f} is a function that takes an element of
type~\mintinline{Ocaml}{a} and returns an element of
type~\mintinline{Ocaml}{b -> c}, \ie a function from~\mintinline{Ocaml}{b}
to~\mintinline{Ocaml}{c}, while~\mintinline{Haskell}{g} is a function that
takes a function from~\mintinline{Ocaml}{a} to~\mintinline{Ocaml}{b} as an
argument an returns an element of type~\mintinline{Ocaml}{c}. If we
regard~\mintinline{Haskell}{->} as an operator, we say it is right associative,
and we can write
\begin{minted}{Haskell}
f :: a -> b -> c
\end{minted}
instead of
\begin{minted}{Haskell}
f :: a -> (b -> c)
\end{minted}

We can use this simulate functions with multiple arguments.
If we consider a function~\mintinline{Haskell}{sum :: Int -> Int -> Int} that
adds two integers and two elements~\mintinline{Haskell}{41 :: Int}
and~\mintinline{Haskell}{19 :: Int} we would have
\begin{minted}{Haskell}
sum       :: Int -> Int -> Int   -- [1]
sum 41    :: Int -> Int          -- [2]
sum 41 19 :: Int                 -- [3]
\end{minted}
where
\begin{enumerate}
    \item The~\mintinline{Haskell}{sum :: Int -> Int -> Int} function takes an
        argument of type~\mintinline{Haskell}{Int} and outputs a function of
        type~\mintinline{Haskell}{Int -> Int}.
    \item After evaluating~\mintinline{Haskell}{sum :: Int -> Int -> Int}
        on~\mintinline{Haskell}{41 :: Int} we get a
        function~\mintinline{Haskell}{Int -> Int},
        denoted~\mintinline{Haskell}{sum 41}.
        Intuitively,~\mintinline{Haskell}{sum 41} is a function that takes an
        integer and returns the result of adding~\(41\) to that integer.
    \item We can evaluate the function~\mintinline{Haskell}{sum 41}
        on~\mintinline{Haskell}{19 :: Int} to get a final result of
        type~\mintinline{Haskell}{Int}.
\end{enumerate}

We can also define operations, or infix functions. To do this we surround the
name of the function with parentheses as such
\begin{minted}{Haskell}
(+) :: Int -> Int -> Int
\end{minted}
and we can later write
\begin{minted}{Haskell}
1 + 2 :: Int
\end{minted}

\subsubsection{Function bodies}
To explain how to define the body of a function we start with an example
\begin{minted}{Haskell}
square :: Int -> Int   -- [1]
square n = n * n       -- [2]
\end{minted}
where
\begin{enumerate}
    \item We first declare that we will define a
        function~\mintinline{Haskell}{square} of
        type~\mintinline{Haskell}{Int -> Int}.
    \item Here we define the behaviour of our function given an
        argument~\mintinline{Ocaml}{n}. The notation is similar to the one
        we use in math, where we write \(f(n) = n*n\).
\end{enumerate}
Notice how the body of the function~\mintinline{Haskell}{square} also has no
parentheses, as in function composition. These are optional again.

Another example
\begin{minted}{Haskell}
multiply :: Int -> Int -> Int
multiply n m = n * m
\end{minted}
although functions in Haskell technically only take one argument, we can define
a function that takes multiple parameters in one line as shown in the example
above.

Definitions like
\begin{minted}{Haskell}
multiplyByTwo :: Int -> Int
multiplyByTwo = multiply 2
\end{minted}
are also valid. Here the function~\mintinline{Haskell}{multiplyByTwo} is defined
in terms of the function~\mintinline{Haskell}{multiply 2 :: Int -> Int}.

We can also define nameless functions by writing
\begin{minted}{Haskell}
\x -> x * x
\end{minted}
where~\mintinline{Haskell}|\| is notation to mean~\(\lambda\). This is useful
in situations where we need a function locally, for example
\begin{minted}{Haskell}
multiplyBy :: Int -> (Int -> Int)
multiplyBy n = \x -> x * n
\end{minted}

Lastly, we define the composition operator. If~\mintinline{Haskell}{f :: a -> b}
and~\mintinline{Haskell}{g :: b -> c} are functions, we can compose them with
the composition operator
\begin{minted}{Ocaml}
g . f :: a -> c
\end{minted}
which is defined as
\begin{minted}{Haskell}
(.) :: (b -> c) -> (a -> b) -> a -> c
(.) g f x = g(f(x))
\end{minted}

\subsubsection{Algebraic data types}
We can also define new types from existing ones. We already know how to do this
with~\mintinline{Haskell}{->}, but we can also use
the~\mintinline{Haskell}{data} keyword
\begin{minted}{Haskell}
data Complex = Cartesian Double Double
\end{minted}
which consists of a space-separated blueprint~\mintinline{Haskell}{Cartesian
Double Double} for the new~\mintinline{Haskell}{Complex} type.

This defines a new type called~\mintinline{Haskell}{Complex}. This definition
comes with an associated
map~\mintinline{Haskell}{Cartesian :: Double -> Double -> Complex}, called
an~\emph{element constructor}, which lets us instantiate elements in
the~\mintinline{Haskell}{Cartesian} type, \ie
\begin{minted}{Haskell}
Cartesian 1.0 2.0 :: Complex
\end{minted}

Here we constructed a type~\mintinline{Haskell}{Complex} from
two~\mintinline{Haskell}{Double} types. We can see this as the Cartesian product
of two types, which in math we would
write~\(\mathbb{C}=\mathbb{R}\times\mathbb{R}\).

We can also define new types in a similar fashion to the disjoint union with
\begin{minted}{Haskell}
data Shape = Circle Double | Polygon Int Double
\end{minted}
where~\mintinline{Haskell}{|} reads as ``or''. Here, a member of
the~\mintinline{Haskell}{Shape} type could be either
a~\mintinline{Haskell}{Circle}, specified by the length of
type~\mintinline{Haskell}{Double} of its radius or a
(regular)~\mintinline{Haskell}{Polygon}, specified the number of sides of
type~\mintinline{Haskell}{Int} and its apothem length, also of
type~\mintinline{Haskell}{Double}.

So far we are capable of defining new types by multiplication and addition. We
call the first kind of types~\emph{product types} and the second type~\emph{sum
types}. Types made combining these operations are called~\emph{algebraic data
types}.

The definition of~\mintinline{Haskell}{Shape} includes two element constructors
\begin{minted}{Haskell}
Circle  :: Double -> Shape
Polygon :: Int -> Double -> Shape
\end{minted}
To implement a function to calculate the area of our shapes we would write
\begin{minted}{Haskell}
area :: Shape -> Double
area (Circle r)    = pi*r*r
area (Polygon n a) = n*a*a*tan(pi/n)
\end{minted}
where we are using the element constructors~\mintinline{Haskell}{Circle}
and~\mintinline{Haskell}{Polygon} for~\emph{pattern matching}. To define
the~\mintinline{Haskell}{area} function on an element we must
first be able to identify what kind of~\mintinline{Haskell}{Shape} it is. This
is done through matching the instance of said element on the element
constructors, and the program will make sure to execute the adequate one.

We use pattern matching to specify the output if the
input~\mintinline{Haskell}{Shape} is of the form~\mintinline{Haskell}{Circle r}
or of the form~\mintinline{Haskell}{Polygon n a}. This must use all of the
constructors found in the~\mintinline{Haskell}{Shape} type declaration for the
function to be non-partial.

Pattern matching also applies to the previous example:
\begin{minted}{Haskell}
conjugate :: Complex -> Complex
conjugate (Cartesian x y) = Cartesian x (-y)
\end{minted}
notice here the use of the~\mintinline{Haskell}{Cartesian} element constructor
to pattern match and to construct an element of
the~\mintinline{Haskell}{Complex} type.

We do not always have to pattern match. We can avoid it when we can treat the
different options equally and we do not need to access the parameters used to
construct the instance of the type, for example
\begin{minted}{Haskell}
sameArea :: Shape -> Shape -> Bool
sameArea s1 s2 = (area s1) == (area s2)
\end{minted}
where~\mintinline{Haskell}{(==)} is the comparison operator, which
outputs~\mintinline{Haskell}{True :: Bool} if its operands are equal,
or~\mintinline{Haskell}{False :: Bool} if they are not.

\subsubsection{Parametrized data types}
So far we have seen how we can define types from other concrete types. The
Haskell language also allows us to define types parametrized by other types, for
example we can define a pair tree type to hold nodes of any type as
\begin{minted}{Haskell}
data Pair a = Tuple a a
\end{minted}

Here~\mintinline{Ocaml}{a} is a generic type, and acts as a parameter to the
type definition. We can then instantiate a pair
of~\mintinline{Haskell}{Int} elements as
\begin{minted}{Haskell}
Tuple 2 3 :: Pair Int
\end{minted}
This definition only allows pairs of elements of the same type. We will
generalize it in a moment.

Examples of generic functions over the type are
\begin{minted}{Haskell}
first :: Pair a -> a
first (Tuple x _) = x

second :: Pair a -> a
second (Tuple _ y) = y

flip :: Pair a -> Pair a
flip (Tuple x y) = Tuple y x

apply :: (a -> a -> b) -> Pair a -> b
apply f (Tuple x y) = f x y
\end{minted}
Notice how to define the~\mintinline{Haskell}{first} function, it is not
necessary to know the value of the second element of the pair, and in fact
remains unused through the body of the function. We mark this with
a~\mintinline{Haskell}{_} character, instead of assigning a variable name to the
argument. The Haskell language requires us to do so, to avoid unused arguments
which are often the source of errors and undefined behaviour in code. The
situation is analogue for the~\mintinline{Haskell}{second} function.

Generic types can be parametrized by more than one type. We can generalize
the~\mintinline{Haskell}{Pair} into
\begin{minted}{Haskell}
data Pair a b = Tuple a b
\end{minted}
which is parametrized by two types, and then we would have
\begin{minted}{Haskell}
Tuple 3 'c' :: Pair Int Char

first :: Pair a b -> a
first (Tuple x _) = x

second :: Pair a b -> b
second (Tuple _ y) = y

flip :: Pair a b -> Pair b a
flip (Tuple x y) = Tuple y x

apply :: (a -> b -> c) -> Pair a b -> c
apply f (Tuple x y) = f x y
\end{minted}

In fact, Haskell already provides a pair data type, the tuple, which is denoted
as~\mintinline{Haskell}{(a, b)}. Given two elements~\mintinline{Haskell}{1 ::
Int} and~\mintinline{Haskell}{'a' :: Char} the corresponding element of the pair
type combining them is also written, somewhat confusingly,
as~\mintinline{Haskell}{(1, 'a') :: (Int, Char)}.

\subsection{The \texorpdfstring{\mintinline{Haskell}{Monad}}{Monad} typeclass}
The Haskell language allows users to define~\emph{typeclasses}, which area set
of constraints we give to a certain set of types.
They are similar to interfaces, which appear in other languages.

For our purposes, we can picture them as a definition. A typeclass describes a
behavior we expect.

Let's see an example. The~\mintinline{Haskell}{Functor} typeclass is
\begin{minted}{Haskell}
class Functor f where
    fmap :: (a -> b) -> f a -> f b
\end{minted}
From the usage of~\mintinline{Ocaml}{f}, the Haskell compiler can infer
that it must be a parametric type with one parameter, and
with this information we can read the previous statement as
\begin{quote}
    For a parametric datatype~\mintinline{Ocaml}{f} with one parameter to be
    called a Functor, it must be equipped with a structure

    \mintinline{Haskell}{fmap :: (a -> b) -> f a -> f b}
\end{quote}
Given a parametric datatype~\mintinline{Ocaml}{f}, the functor typeclass
provides a function that can be understood as having
type~\mintinline{Haskell}{fmap :: (a -> b) -> (f a -> f b)}, which reminds us of
the property of functors to act on functions. Additionally, the parametric
datatype itself acts on types, by transforming an existing type into a new one,
behaviour similar to that of a functor.

Let's see an example of an implementation. If we define the parametric data
type~\mintinline{Haskell}{Maybe} as follows
\begin{minted}{Haskell}
data Maybe a = Just a | None
\end{minted}
Notice~\mintinline{Haskell}{None} is a singleton.
We can make~\mintinline{Haskell}{Maybe} into a Functor by writing
\begin{minted}{Haskell}
instance Functor Maybe where
    fmap :: (a -> b) -> Maybe a -> Maybe b
    fmap _ Nothing  = Nothing
    fmap f (Just x) = Just (f x)
\end{minted}
where we first declare that we want to show that~\mintinline{Haskell}{Maybe} is
a~\mintinline{Haskell}{Functor}, which we do by providing a definition
for~\mintinline{Haskell}{fmap :: (a -> b) -> Maybe a -> Maybe b}.

This is not enough for a function on objects and morphisms to be a functor. The
functor laws must be satisfied as well. The Haskell language does not provide a
way to express such laws, since for this it would need a way to write and
validate proofs, which would add too much complexity to the language. The
compiler assumes any typeclass implementations it encounters satisfy such
properties, which are only generally found in external documentation. Failing to
satisfy these laws could lead to buggy code and unintended behaviour.

We will not prove typeclass laws for specific instances we provide in this
section, as the proofs are usually straightforward but uncomfortable to write.

The laws for the functor typeclass can be written as
\begin{minted}{Ocaml}
fmap id = id
fmap (f . g) = (fmap f . fmap g)
\end{minted}
where an informal mix of mathematical and Haskell notation is typically used.
Most importantly, the~\mintinline{Ocaml}{=} in this case refers to the
mathematical equality.

Notice that these laws are clearly equivalent to the functor laws found
in~\ref{def:functor}.  The~\mintinline{Ocaml}{f} that appears in these laws is
different from the~\mintinline{Ocaml}{f} found in the definition of
the~\mintinline{Haskell}{Functor} typeclass, since it now corresponds to an
arbitrary function. This misleading use of notation happens often in Haskell,
and will be repeated in further typeclass laws in this document.

There is a notion of~\emph{inheritance} of typeclasses. Typeclass inheritance is
when a typeclass has a superclass. This is a way of expressing that a typeclass
requires another typeclass to be available for a given type before you can write
an instance. For example, Haskell provides a class that inherits
from~\mintinline{Haskell}{Functor}
\begin{minted}{Haskell}
class Functor f => Applicative f where
    (<*>) :: f (a -> b) -> f a -> f b
    pure :: a -> f a
\end{minted}
the~\mintinline{Haskell}{=>} notation is unfortunate, as in math it is usually
associated with an implication.

In this situation, to prove that a datatype is
an~\mintinline{Haskell}{Applicative}, we must first prove it is
a~\mintinline{Haskell}{Functor}, and then provide implementations for the two
required functions.
The laws for the~\mintinline{Haskell}{Applicative} typeclass are
\begin{minted}{Haskell}
pure id <*> v = v
pure (.) <*> u <*> v <*> w = u <*> (v <*> w)
pure f <*> pure x = pure (f x)
u <*> pure y = pure (\f -> f y) <*> u
\end{minted}
and we can make~\mintinline{Haskell}{Maybe}, which we know to be
a~\mintinline{Haskell}{Functor}, into an~\mintinline{Haskell}{Applicative}
\begin{minted}{Haskell}
instance Applicative Maybe where
    (<*>) :: Maybe (a -> b) -> Maybe a -> Maybe b
    (<*>) (Just f) m = fmap f m
    (<*>) Nothing  _ = Nothing

    pure :: a -> Maybe a
    pure = Just
\end{minted}
For our interests, this type class is only used as an intermediary building
block to then define the~\mintinline{Haskell}{Monad} typeclass, which is defined
as
\begin{minted}{Haskell}
class Applicative m => Monad m where
    (>>=) :: m a -> (a -> m b) -> m b
    (>>) :: m a -> m b -> m b
    return :: a -> m a
\end{minted}
where~\mintinline{Haskell}{(>>=)} is called~\emph{bind}.
Any~\mintinline{Haskell}{Monad} implementation must satisfy
\begin{minted}{Haskell}
return a >>= k = k a
m >>= return = m
m >>= (\x -> k x >>= h) = (m >>= k) >>= h

k >> f = k >>= \_ -> f

pure = return
m1 <*> m2 = m1 >>= (x1 -> m2 >>= (x2 -> return (x1 x2)))
\end{minted}
finally, we can make~\mintinline{Haskell}{Maybe}, which we know to be
an~\mintinline{Haskell}{Applicative}, into a~\mintinline{Haskell}{Monad}
\begin{minted}{Haskell}
instance Monad Maybe where
    (>>=) :: Maybe a -> (a -> Maybe b) -> Maybe b
    (Just x) >>= k = k x
    Nothing  >>= _ = Nothing

    (>>) :: Maybe a -> Maybe b -> Maybe b
    (Just _) >> m = m
    Nothing  >> _ = Nothing

    return :: a -> Maybe a
    return = pure
\end{minted}

\subsection{Haskell monads are monads}
The~\mintinline{Haskell}{Monad} structure defined by Haskell
\begin{minted}{Haskell}
class Applicative m => Monad m where
    (>>=) :: m a -> (a -> m b) -> m b
    (>>) :: m a -> m b -> m b
    return :: a -> m a
\end{minted}
subject to the laws
\begin{minted}{Haskell}
return a >>= k = k a
m >>= return = m
m >>= (\x -> k x >>= h) = (m >>= k) >>= h

k >> f = k >>= \_ -> f

pure = return
m1 <*> m2 = m1 >>= (x1 -> m2 >>= (x2 -> return (x1 x2)))
\end{minted}
does not correspond, at first sight, to the definition of monad given
in~\ref{def:monad}.

We begin observing that the~\mintinline{Haskell}{Monad}
operation~\mintinline{Haskell}{(>>)} can be uniquely determined in terms
of~\mintinline{Haskell}{(>>=)}, the bind operation.
This means that we can omit this operation from our reasoning. In fact, it is
only defined because it is useful in a programming setting, and bears no
particular mathematical relevance to us.

Our next observation concerns the bind operation. If we flip its arguments, we
get an operation with type signature
\begin{minted}{Haskell}
(a -> m b) -> (m a -> m b)
\end{minted}
which coincides with the~\((-)^{\ast}\) operation, while the type signature for
the~\mintinline{Haskell}{return} function coincides with~\(\eta_{A}\)
(where~\(A\) is determined by context in the Haskell program), as defined in the
definition of Kleisli triple in~\ref{def:kleisli-triple}.

With this correspondence we can now prove that the~\mintinline{Haskell}{Monad}
structure defined by Haskell is a monad by first showing that it is a Kleisli
triple, and then using
theorem~\ref{thm:kleisli-triples-and-monads-correspondence}. This can be shown
directly from the first three~\mintinline{Haskell}{Monad} laws.

We now have the tools to implement the notions we introduced in
section~\ref{sec:kleisli-triple}.

In fact, we have already covered on example~\ref{ex:kleisli-partiality}, which
was on partiality. It corresponds to the~\mintinline{Haskell}{Maybe} monad.
The~\(\bot\) element corresponds to~\mintinline{Haskell}{Nothing}, and the
implementation of the~\mintinline{Haskell}{(>>=)} operator is identical to how
we define~\((-)^{\ast}\) in the example.

We can directly implement the Kleisli triples we saw in the examples in the
beginning of section~\ref{sec:kleisli-triple}, in a similar fashion to
the~\mintinline{Haskell}{Maybe} data type.

\subsection{Other monads}
Although the~\mintinline{Haskell}{Monad} typeclass is called monad, the
traditional definition of monads, as shown in~\ref{def:monad}, does not
explicitly appear anywhere in Haskell so far. The~\mintinline{Haskell}{Monad}
typeclass itself is based on the definition for a Kleisli
triple~\ref{def:kleisli-triple}.

This begs the question on why the~\mintinline{Haskell}{Monad} typeclass is
called monad. In~\cite{Moggi-notions-computation-monads} Moggi mentions monads
are mode widely used in the literature on Category Theory, and have the
advantage of being defined only in terms of functors and natural
transformations, which make them more suitable for abstract manipulation.

Monads are also more prevalent in computer science. Many of the structures that
computer scientists work with are monadic in some sense, as can be seen
in~\cite{hackage-monad}. This means that the introduction of
the~\mintinline{Haskell}{Monad} abstraction to the language did not, in theory,
increase the complexity of the field, as it was already a member of it.

\subsection{The \texorpdfstring{\mintinline{Haskell}{IO}}{IO} monad and impurity}
We introduce one of the most important monads in Haskell,
the~\mintinline{Haskell}{IO} monad, short of input-output, which enables Haskell
programmers to perform system input and output operations in a pure context.

This is based on the side effects Kleisli triple~\ref{ex:kleisli-side-effects}.
In favour of not introducing yet more notation, we assume we have a
type~\mintinline{Haskell}{World} containing some data about the state of the
computer. We define the~\mintinline{Haskell}{IO} data type to be
\begin{minted}{Haskell}
data IO a = IO World -> (World, a)
\end{minted}
We will model input and output operations through this data type. For example, a
function that reads an~\mintinline{Haskell}{Int} from user input would have type
\begin{minted}{Haskell}
readNumber :: IO Int
\end{minted}
which can be understood as
\begin{minted}{Haskell}
readNumber :: World -> (World, Int)
\end{minted}
We should interpret this as~\mintinline{Haskell}{readNumber} being a function
that, when run, considers the state of the computer and returns a
pair~\mintinline{Haskell}{(World, Int)} containing the state of the computer
after having read a~\mintinline{Haskell}{Int} from it (might be unmodified),
and the~\mintinline{Haskell}{Int} itself.

Another example would be a function that stores an~\mintinline{Haskell}{Int} to
a file. It would have type
\begin{minted}{Haskell}
saveNumber :: Int -> IO ()
\end{minted}
where~\mintinline{Haskell}{()} is the empty type, defined
as~\mintinline{Haskell}{data () = ()}. Again, the type signature can be
understood as
\begin{minted}{Haskell}
saveNumber :: Int -> World -> (World, ())
\end{minted}
We should interpret~\mintinline{Haskell}{saveNumber} as a function that receives
a number and the state of the computer, and returns the updated state after
having saved the number to a file (a state with the file containing the number),
and no other result, as we did not request any additional information.

This approach begs the question: how is handling the entire state of the
computer, from inside the computer itself, viable? Just passing or returning the
entire memory state to or from a function would require the same amount of
memory to store the data. To answer this we must first narrow down our problem.

Traditional computers are state machines~\cite{myers-state-machines}, which are
highly stateful, and arguably impure code by design. We will avoid dwelling on
this topic as it is not directly related to our work, and bordering the
philosophical. For example, the argument that computers increase the entropy of
the universe by releasing heat when running code, and as such are impure because
that is a side-effect, can be made.

For this we make the distinction to only refer to the paradigm of a programming
language up to compilation, the process of converting the written code to
something a machine understands (roughly speaking), since once they are compiled
they must be shaped into stateful and impure code that a machine will execute.
The advantages of pure functions are thanks to the reasoning the compiler can
apply on them. As such, this distinction is justified.

Following this, we say that a language is purely functional if the compiler is
designed to only deal with pure functions, and the abstractions we defined to
ensure these functions are pure are only used to enable the compiler to reason
about our code and produce machine code that satisfies some guarantees, while
being itself impure.
\end{document}
